%%
%% This is file `xampl-ch1.tex',
%% generated with the docstrip utility.
%%
%% The original source files were:
%%
%% vakthesis.dtx  (with options: `xampl-ch1')
%% 
%% IMPORTANT NOTICE:
%% 
%% For the copyright see the source file.
%% 
%% Any modified versions of this file must be renamed
%% with new filenames distinct from xampl-ch1.tex.
%% 
%% For distribution of the original source see the terms
%% for copying and modification in the file vakthesis.dtx.
%% 
%% This generated file may be distributed as long as the
%% original source files, as listed above, are part of the
%% same distribution. (The sources need not necessarily be
%% in the same archive or directory.)
%% xampl-ch1.tex  Приклад розділу дисертації
% Приклад назви розділу і мітки, на яку можна посилатися в тексті
\chapter{Подання дійсних чисел рядами~Остроградського $1$-го виду}
\label{ch:o1series}

Це не є справжній розділ дисертації. Це лише приклад, який повинен
допомогти користувачу підготувати свій файл. Але я зробив його з
розділу~1 своєї дисертації.

У цьому розділі вивчається розвинення дійсного числа у
знакозмінний ряд спеціального вигляду, який називається рядом
Остроградського $1$-го виду.

Досліджуються тополого"=метричні та фрактальні властивості множини
неповних сум заданого ряду Остроградського $1$-го виду, а також
властивості розподілів ймовірностей на множині неповних сум.


% Приклад назви підрозділу
\section{Означення ряду Остроградського $1$-го виду}

% Приклад означення
\begin{definition}
% Приклад виноски (\footnote)
\emph{Рядом Остроградського $1$-го виду}\footnote{Далі часто
будемо називати просто \emph{рядом Остроградського}, оскільки ми
не досліджуємо ряди Остроградського $2$-го виду.} називається
скінченний або нескінченний вираз вигляду
% Приклад формули з номером
\begin{equation}\label{eq:o1series}
\frac1{q_1}-\frac1{q_1q_2}+\dots +\frac{(-1)^{n-1}}{q_1q_2\dots
q_n}+\dotsb,
\end{equation}
де $q_n$ "--- натуральні числа і $q_{n + 1}>q_n$ для будь-якого
$n\in\N$. Числа $q_n$ називаються \emph{елементами ряду
Остроградського $1$-го виду}.
\end{definition}

Число елементів може бути як скінченним, так і нескінченним.
У~першому випадку будемо записувати ряд Остроградського у вигляді
% Приклад формули без номера
\[
\frac1{q_1}-\frac1{q_1q_2}+\dots +\frac{(-1)^{n-1}}{q_1q_2\dots
q_n}
\]
або скорочено
\begin{equation*}
\Osign1(q_1,q_2,\dots,q_n)
\end{equation*}
і називати скінченним рядом Остроградського або
$n$\nobreakdash-\hspace{0pt}елементним рядом Остроградського; а в
другому випадку будемо записувати ряд Остроградського у вигляді
% Приклад посилання на формулу
\eqref{eq:o1series} або скорочено
\begin{equation*}
\Osign1(q_1,q_2,\dots,q_n,\dots)
\end{equation*}
і називати нескінченним рядом Остроградського.


\section{Означення та властивості підхідних чисел}

\begin{definition}\label{def:convergent}
\emph{Підхідним числом порядку $k$} ряду Остроградського $1$-го
виду називається раціональне число
\[
\frac{A_k}{B_k} = \frac{1}{q_1}-\frac{1}{q_1q_2}+\dots
+\frac{(-1)^{k-1}}{q_1q_2\dots q_k} = \Osign1(q_1,q_2,\dots,q_k).
\]
\end{definition}

Зрозуміло, що $n$-елементний ряд Остроградського має $n$ підхідних
чисел, причому підхідне число $n$-го порядку $\frac{A_n}{B_n}$
збігається зі значенням цього ряду Остроградського.

% Приклад теореми
\begin{theorem}\label{th:convergents}
Для будь-якого натурального $k$ правильні формули
\begin{equation}\label{eq:convergents}
\left\{
\begin{aligned}
&A_k=A_{k-1}q_k+(-1)^{k-1},\\
&B_k=B_{k-1}q_k=q_1q_2\dots q_k
\end{aligned}
\right.
\end{equation}
\textup(якщо покласти, що $A_0=0$, $B_0=1$\textup).
\end{theorem}

% Приклад доведення
\begin{proof}
Проведемо доведення методом математичної індукції по~$k$. Для
$k=1$ формули правильні. Справді,
\[
\frac{A_1}{B_1}=\frac{1}{q_1}=\frac{A_0q_1+(-1)^0}{B_0q_1}.
\]

Припустимо, що формули \eqref{eq:convergents} правильні для
деякого $k=m$, тобто
\begin{align*}
\left\{
\begin{aligned}
&A_m=A_{m-1}q_m+(-1)^{m-1},\\
&B_m=B_{m-1}q_m=q_1q_2\ldots q_m,
\end{aligned}
\right.
\end{align*}
і доведемо ці формули для $k=m+1$. Маємо
% Приклад формули, що займає більше одного рядка.
% Оточення align, рядки вирівняні по знаку =
\begin{align*}
\frac{A_{m+1}}{B_{m+1}}&=\frac{1}{q_1}-\frac{1}{q_1 q_2}+\dots
+\frac{(-1)^{m-1}}{q_1q_2\dots q_m}+\frac{(-1)^m}{q_1q_2\dots
q_mq_{m + 1}}=\\ &=\frac{A_m}{B_m}+\frac{(-1)^m}{B_mq_{m + 1}} =
\frac{A_mq_{m+1}+(-1)^m}{B_mq_{m + 1}}.
\end{align*}

Отже, за принципом математичної індукції формули
\eqref{eq:convergents} правильні для будь"=якого натурального $k$.
\end{proof}

% Приклад леми
\begin{lemma}
Для будь-якого натурального $k$ правильна рівність
\begin{equation}\label{eq:convergents1}
\frac{A_{k-1}}{B_{k-1}}-\frac{A_k}{B_k}=\frac{(-1)^k}{B_k}.
\end{equation}
\end{lemma}

\begin{lemma}
Для будь-якого натурального $k\geq2$ правильна рівність
\begin{equation}\label{eq:convergents2}
\frac{A_{k-2}}{B_{k-2}}-\frac{A_k}{B_k}=\frac{(-1)^{k-1}(q_k-1)}{B_k}.
\end{equation}
\end{lemma}


\section{Розклад числа у знакозмінний ряд за $1$-м алгоритмом Остроградського}

Почнемо з геометричної ілюстрації алгоритму. Нехай маємо відрізки
$A$ та $B$, $A<B$. Щоб застосувати $1$-й алгоритм Остроградського
до числа $\frac{A}{B}$, будемо відкладати відрізок $A$ на відрізку
$B$, поки не отримаємо залишок $A_1<A$ (див.
% Приклад посилання на малюнок
рис.~\ref{fig:o1alg}). Нехай відрізок $A$ вміщується $q_1$ разів у
відрізку $B$, тоді
\[
B=q_1A+A_1.
\]
Далі відкладемо відрізок $A_1$ не на меншому відрізку $A$ (як у
алгоритмі Евкліда), а на тому ж відрізку $B$ до отримання залишку
$A_2<A_1$. Нехай відрізок $A_1$ вміщується $q_2$ разів у відрізку
$B$, тоді
\[
B=q_2A_1+A_2.
\]
Відкладаючи відрізок $A_2$ знову на відрізку $B$ і~т.~д. до
нескінченності або до отримання нульового залишку, будемо мати
\begin{align*}
&B=q_3A_2+A_3,\\
&B=q_4A_3+A_4
\end{align*}
і~т.~д. З отриманих рівностей випливає, що має місце розклад
\[
\frac AB = \frac{1}{q_1} - \frac{1}{q_1q_2} + \frac{1}{q_1q_2q_3}
- \frac{1}{q_1q_2q_3q_4} + \dotsb,
\]
і тут, як легко бачити,
\[
q_1<q_2<q_3<q_4<\dotsb.
\]

% Приклад малюнка
\begin{figure}[htbp]
\setlength{\unitlength}{1mm}
\begin{center}
\begin{picture}(105,35)
\put(6,28){$A$}\put(51,28){$A_1$} \put(0,25){\line(1,0){58}}
\multiput(0,24)(16,0){4}{\line(0,1){2}}\put(58,24){\line(0,1){2}}
\put(28,19){$B$} \put(75,25){$B=3A+A_1$}
\put(3,9){$A_1$}\put(52,9){$A_2$} \put(0,6){\line(1,0){58}}
\multiput(0,5)(10,0){6}{\line(0,1){2}}\put(58,5){\line(0,1){2}}
\put(28,0){$B$} \put(75,6){$B=5A_1+A_2$}
\end{picture}
\caption{Геометрична ілюстрація $1$-го алгоритму Остроградського:
тут відрізок $A$ вміщується 3~рази у відрізку $B$, відрізок $A_1$
вміщується 5~разів у відрізку $B$ і~т.~д.}
\label{fig:o1alg}
\end{center}
\end{figure}

Таким чином, \emph{$1$-й алгоритм Остроградського} розкладу
дійсного числа $x\in(0,1)$ у знакозмінний ряд полягає в
наступному.

\begin{description}
\item[Крок~1.] Покласти $\alpha_0=x$, $i=1$.

\item[Крок~2.] Знайти такі числа $q_i$ та $\alpha_i$, що
\[
1=q_i\alpha_{i-1}+\alpha_i \quad \text{і} \quad
0\leq\alpha_i<\alpha_{i-1}.
\]

\item[Крок~3.] Якщо $\alpha_i=0$, то припинити обчислення. Інакше
"--- збільшити $i$ на $1$ та перейти до кроку~2.
\end{description}

\begin{theorem}\label{thm:ostrogradsky}
Кожне дійсне число $x\in(0,1)$ можна подати у вигляді ряду
Остроградського $1$-го виду~\eqref{eq:o1series}. Причому, якщо
число $x$ ірраціональне, то це можна зробити єдиним чином і
вираз~\eqref{eq:o1series} має при цьому нескінченне число
доданків; якщо ж число $x$ раціональне, то його можна подати у
вигляді~\eqref{eq:o1series} зі скінченним числом доданків двома
різними способами:
\[
x=\Osign1(q_1,q_2,\dots,q_{n-1},q_n)=\Osign1(q_1,q_2,\dots,q_{n-1},q_n-1,q_n).
\]
\end{theorem}

% Приклад посилання на таблицю
У табл.~\ref{tab:ellipse.hyperbola.parabola} наведені деякі
формули для еліпса, гіперболи і параболи.

% Приклад таблиці
\begin{table}[htbp]
\caption{Еліпс, гіпербола і парабола. Деякі формули}
\label{tab:ellipse.hyperbola.parabola}
\begin{tabularx}{\textwidth}{|X|c|c|c|}
\hline
                   & Еліпс                                    & Гіпербола                                & Парабола          \\
\hline
Канонічне рівняння & $\frac{x^2}{a^2}+\frac{y^2}{b^2}=1$      & $\frac{x^2}{a^2}-\frac{y^2}{b^2}=1$      & $y^2=2px$         \\
Ексцентриситет     & $\varepsilon=\sqrt{1-\frac{b^2}{a^2}}<1$ & $\varepsilon=\sqrt{1+\frac{b^2}{a^2}}>1$ & $\varepsilon=1$   \\
Фокуси             & $(a\varepsilon,0)$, $(-a\varepsilon,0)$  & $(a\varepsilon,0)$, $(-a\varepsilon,0)$  & $(\frac{p}{2},0)$ \\
\hline
\multicolumn{4}{|l|}{Корн~Г., Корн~Т. Справочник по математике. М., 1974. С.~72.} \\
\hline
\end{tabularx}
\end{table}


\section{Множина неповних сум ряду Остроградського та розподіли ймовірностей на ній}

Візьмемо довільну \emph{фіксовану} послідовність $\{q_k\}$
натуральних чисел з умовою $q_{k+1}>q_k$ для всіх $k\in\N$ і
розглянемо їй відповідний ряд Остроградського $1$-го
виду~\eqref{eq:o1series} з сумою $r$. Число $r$ можна записати у
вигляді
\begin{equation}
r=d-b, \quad \text{де} \quad d=\sum_{i=1}^\infty
\frac1{q_1q_2\dots q_{2i-1}}, \quad b=\sum_{i=1}^\infty
\frac1{q_1q_2\dots q_{2i}}.
\end{equation}

% Приклад назви пункту
\subsection{Тополого-метричні та фрактальні властивості множини
неповних сум ряду Остроградського}

\emph{Циліндром} рангу $m$ з основою $c_1c_2\dots c_m$ називається
множина $\Delta'_{c_1c_2\dots c_m}$ всіх неповних сум, які мають
зображення $\Delta_{c_1c_2\dots c_ma_{m+1}\dots a_{m+k}\dots}$, де
$a_{m+j}\in\set{0,1}$ для будь-якого $j\in\N$. Очевидно, що
\[
\Delta'_{c_1c_2\dots c_ma}\subset\Delta'_{c_1c_2\dots c_m}, \quad
a\in\set{0,1}.
\]

% Приклад теоремоподібної структури з додатковою інформацією в заголовку
\begin{definition}[{\cite[с.~59]{Pra98}}]
\emph{Фракталом} називається кожна континуальна обмежена множина
простору $\R^1$, яка має тривіальну (рівну $0$ або $\infty$)
$H_\alpha$-міру Хаусдорфа, порядок $\alpha$ якої дорівнює
топологічній розмірності.
\end{definition}

Ті нуль-множини Лебега простору $\R^1$, розмірність
Хаусдорфа\nobreakdash--\hspace{0pt}Безиковича яких дорівнює $1$,
називаються \emph{суперфракталами}, а континуальні множини, що
мають нульову розмірність Хаусдорфа--Безиковича, називаються
\emph{аномально фрактальними}.


% Приклад висновків до розділу
\section*{Висновки до розділу~\ref{ch:o1series}}

У розділі~\ref{ch:o1series} введене поняття ряду Остроградського
$1$-го виду та його підхідних чисел, запропоновані деякі
властивості підхідних чисел. Доведено, що кожне дійсне число
$x\in(0,1)$ можна подати у вигляді ряду Остроградського $1$-го
виду: ірраціональне "--- єдиним чином у вигляді нескінченного ряду
Остроградського, раціональне "--- двома різними способами у
вигляді скінченного ряду Остроградського. Ці результати не є
новими, їх можна знайти, наприклад, у
роботах~\cite{Rem51,Sie11STNW,Pie29,VaZ75,Sha86} та~ін. Вони
наведені тут для повноти викладу.

Новими в цьому розділі є результати, що стосуються неповних сум
ряду Остроградського. Описані тополого-метричні та фрактальні
властивості множини неповних сум ряду Остроградського. Описано
множини чисел, ряди Остроградського яких є простими і густими
відповідно. Доведено, що випадкова неповна сума ряду
Остроградського має або дискретний розподіл або сингулярний
розподіл канторівського типу. Досліджено поведінку на
нескінченності модуля характеристичної функції випадкової неповної
суми ряду Остроградського.
