%%
%% This is file `xampl-concl.tex',
%% generated with the docstrip utility.
%%
%% The original source files were:
%%
%% vakthesis.dtx  (with options: `xampl-concl')
%% 
%% IMPORTANT NOTICE:
%% 
%% For the copyright see the source file.
%% 
%% Any modified versions of this file must be renamed
%% with new filenames distinct from xampl-concl.tex.
%% 
%% For distribution of the original source see the terms
%% for copying and modification in the file vakthesis.dtx.
%% 
%% This generated file may be distributed as long as the
%% original source files, as listed above, are part of the
%% same distribution. (The sources need not necessarily be
%% in the same archive or directory.)
%% xampl-concl.tex  Приклад висновків до дисертації
\chapter*{Висновки}

Це не є справжні висновки до дисертації. Це лише приклад, який
повинен допомогти користувачу підготувати свій файл. Але я зробив
його з висновків до своєї дисертації.

Ряди Остроградського $1$-го виду дозволяють розширити можливості
формального задання і аналітичного дослідження фрактальних множин,
сингулярних мір, недиференційовних функцій та інших об'єктів зі
складною локальною будовою.

В дисертаційній роботі отримано такі результати.
\begin{itemize}
\item Розроблено основи метричної теорії чисел, представлених
рядами Остроградського $1$-го виду. Зокрема, досліджено геометрію
розвинень чисел в ряди Остроградського $1$-го виду, отримано
основне метричне відношення та його оцінки, які допомагають у
розв'язанні задач про міру Лебега множин чисел з умовами на
елементи зображення.

\item Знайдено умови нуль-мірності (додатності міри) певних класів
замкнених ніде не щільних множин чисел, заданих умовами на
елементи їх розвинення в ряд Остроградського $1$-го виду.

\item Вивчено тополого-метричні та фрактальні властивості множини
неповних сум заданого ряду Остроградського $1$-го виду та
розподілів ймовірностей на ній.

\item Досліджено структуру та властивості випадкової величини з
незалежними різницями послідовних елементів її представлення рядом
Остроградського $1$-го виду.

\item Вивчено диференціальні та фрактальні властивості однієї
функції, заданої перетворювачем елементів ряду Остроградського
$1$-го виду її аргумента в двійкові цифри значення функції.
\end{itemize}

Як виявилося, існують принципові відмінності метричної теорії
рядів Остроградського та метричної теорії ланцюгових дробів.
Зокрема, існує клас замкнених ніде не щільних множин додатної міри
Лебега, описаних в термінах елементів ряду Остроградського. В той
же час, аналогічні множини, задані у термінах елементів
ланцюгового дробу, мають нульову міру Лебега.

Проведені дослідження лежать в руслі сучасних математичних
досліджень об'єктів зі складною локальною поведінкою (будовою),
пов'язаних з ланцюговими дробами, рядами Люрота,
$\beta$-розкладами тощо, інтерес до яких у світі достатньо
високий. Отримані результати та запропоновані методи можуть бути
корисними при розв'язанні задач метричної теорії чисел,
представлених рядами Остроградського $2$-го виду або іншими
зображеннями з нескінченним алфавітом.
