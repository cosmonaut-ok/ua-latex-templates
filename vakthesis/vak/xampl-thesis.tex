%%
%% This is file `xampl-thesis.tex',
%% generated with the docstrip utility.
%%
%% The original source files were:
%%
%% vakthesis.dtx  (with options: `xampl-thesis')
%% 
%% IMPORTANT NOTICE:
%% 
%% For the copyright see the source file.
%% 
%% Any modified versions of this file must be renamed
%% with new filenames distinct from xampl-thesis.tex.
%% 
%% For distribution of the original source see the terms
%% for copying and modification in the file vakthesis.dtx.
%% 
%% This generated file may be distributed as long as the
%% original source files, as listed above, are part of the
%% same distribution. (The sources need not necessarily be
%% in the same archive or directory.)
%% xampl-thesis.tex  Приклад головного файла дисертації
\documentclass{vakthesis}
% Існують кілька опцій, які необхідно вказувати як факультативний
% аргумент команди \documentclass. Наприклад, для докторської
% дисертації необхідно написати
% \documentclass[d]{vakthesis}

% Налагодження кодування шрифта, кодування вхідного файла
% та вибір необхідних мов
\usepackage[T2A]{fontenc}
\usepackage[cp1251]{inputenc}
\usepackage[english,russian,ukrainian]{babel}

% Підключення необхідних пакетів. Наприклад,
% Пакети AMS для підтримки математики, теорем, спеціальних шрифтів
\usepackage[intlimits]{amsmath}
\allowdisplaybreaks
\usepackage{amsthm}
\usepackage{amssymb}
% Налагодження нумерованих списків
%\usepackage{enumerate}
% Гіпертекстові документи
%\usepackage{hyperref}
% або лише спеціальне форматування URL
%\usepackage{url}
% У списку літератури зворотні вказівки на посилання
%\usepackage{backref}
% Сортування посилань
%\usepackage[noadjust]{cite}
% Останні два пакети несумісні між собою. Крім того, конфліктують з цим класом!
% Таблиці зі стовпчиками, що розтягуються
\usepackage{tabularx}

% Налагодження параметрів сторінки (зокрема берегів).
% Наприклад, за допомогою пакета geometry
\usepackage{geometry}
\geometry{hmargin={30mm,15mm},lines=29,vcentering}

% Означення теорем (теоремоподібних структур)
% Класичний варіант: для кожної теореми свій лічильник,
% тобто теорема 1.1, лема 1.1, теорема 1.2
\theoremstyle{plain}
\newtheorem{theorem}{Теорема}[chapter]
\newtheorem{lemma}{Лема}[chapter]
\newtheorem{corollary}{Наслідок}[chapter]
\theoremstyle{definition}
\newtheorem{definition}{Означення}[chapter]
\newtheorem{example}{Приклад}[chapter]
\theoremstyle{remark}
\newtheorem{remark}{Зауваження}[chapter]
% Цікавий варіант: всі теореми нумеруються одним лічильником,
% тобто теорема 1.1, лема 1.2, теорема 1.3
%\theoremstyle{plain}
%\newtheorem{theorem}{Теорема}[chapter]
%\newtheorem{lemma}[theorem]{Лема}
%\newtheorem{corollary}[theorem]{Наслідок}
%\theoremstyle{definition}
%\newtheorem{definition}[theorem]{Означення}
%\newtheorem{example}[theorem]{Приклад}
%\theoremstyle{remark}
%\newtheorem{remark}[theorem]{Зауваження}

% Локальні означення
\newcommand{\N}{\mathbb{N}}
\newcommand{\Z}{\mathbb{Z}}
\newcommand{\Q}{\mathbb{Q}}
\newcommand{\R}{\mathbb{R}}
\newcommand{\set}[1]{\left\{#1\right\}}
\newcommand{\abs}[1]{\left\lvert#1\right\rvert}
\newcommand{\norm}[2][]{\left\lVert#2\right\rVert_{#1}}
% Це потрібно для скороченого запису ряду Остроградського
\newcommand{\Osign}[1]{\mathrm{O}^{#1}}

% Якщо потрібно працювати лише з деякими розділами
%\includeonly{xampl-ch1,xampl-bib}

% Інформація про використані пакети тощо.
% Може знадобитися для відлагодження класу документа
%\listfiles

\begin{document}

% Назва дисертації
\title{Метрична та ймовірнісна теорія чисел,
       представлених рядами Остроградського 1-го~виду}
% Прізвище, ім'я, по батькові здобувача
\author{Барановський Олександр Миколайович}
% Прізвище, ім'я, по батькові наукового керівника/консультанта
\supervisor{Працьовитий Микола Вікторович}
% Науковий ступінь, вчене звання наукового керівника/консультанта
           {доктор фізико-математичних наук, професор}
% Спеціальність
\speciality{01.01.01}
% Варіант із вказуванням факультативних аргументів
%\speciality[математичний аналіз]{01.01.01}[фізико-математичних наук]
% Індекс за УДК
\udc{511.72}
% Установа, де виконана робота, і місто
\institution{Національний педагогічний університет імені~М.~П.~Драгоманова}{Київ}
% Рік, коли написана дисертація
\date{2006}

% Тут буде титульна сторінка
\maketitle

% Зміст
\tableofcontents

% Розділи дисертації в окремих файлах
%%
%% This is file `xampl-intro.tex',
%% generated with the docstrip utility.
%%
%% The original source files were:
%%
%% vakthesis.dtx  (with options: `xampl-intro')
%% 
%% IMPORTANT NOTICE:
%% 
%% For the copyright see the source file.
%% 
%% Any modified versions of this file must be renamed
%% with new filenames distinct from xampl-intro.tex.
%% 
%% For distribution of the original source see the terms
%% for copying and modification in the file vakthesis.dtx.
%% 
%% This generated file may be distributed as long as the
%% original source files, as listed above, are part of the
%% same distribution. (The sources need not necessarily be
%% in the same archive or directory.)
%% xampl-intro.tex  Приклад вступу до дисертації
% Приклад ненумерованого розділу
\chapter*{Вступ}


\paragraph{Актуальність теми}

Це не є справжня дисертація. Це лише приклад, який повинен
допомогти користувачу підготувати свій файл. Але я зробив його із
своєї дисертації. Тому формули, теореми, доведення, імена, книги і
статті у списку літератури інколи можуть бути справжніми (хоча
можуть здаватися безглуздими, бо вирвані з контексту).


\paragraph{Зв'язок роботи з науковими програмами, планами, темами}

Робота виконана у рамках досліджень математичних об'єктів зі
складною локальною будовою, що проводяться на кафедрі вищої
математики Національного педагогічного університету імені
М.",П.",Драгоманова.


\paragraph{Мета і завдання дослідження}

Метою роботи є розробка основ метричної теорії дійсних чисел,
представлених рядами Остроградського $1$-го виду, та застосування
отриманих результатів до дослідження математичних об'єктів зі
складною локальною будовою (фрактальних множин, сингулярних та
недиференційовних функцій, сингулярно неперервних мір).

\subparagraph{Методи дослідження}

У роботі використовувалися методи математичного аналізу, теорії
функцій дійсної змінної, теорії міри, метричної теорії чисел,
теорії ймовірностей, фрактального аналізу тощо.


\paragraph{Наукова новизна одержаних результатів}

Основними науковими результатами, що виносяться на захист, є такі:
% Приклад ненумерованого списку
\begin{itemize}
\item Доведено, що множина неповних сум ряду Остроградського
$1$-го виду є ніде не щільною досконалою множиною нульової міри
Лебега та нульової розмірності Хаусдорфа--Безиковича.

\item Знайдено умови нуль-мірності (додатності міри) певних класів
замкнених ніде не щільних множин чисел, заданих умовами на
елементи їх розвинення в ряд Остроградського $1$-го виду.

\item \ldots
\end{itemize}


\paragraph{Практичне значення одержаних результатів}

Робота має теоретичний характер. Отримані результати є безперечним
внеском у теорію міри, метричну теорію чисел, теорію функцій
дійсної змінної та теорію сингулярних розподілів ймовірностей.
Запропоновані в дисертації методи можуть бути корисними при
дослідженні математичних об'єктів зі складною локальною будовою,
заданих за допомогою інших представлень чисел з нескінченним
алфавітом, зокрема рядів Остроградського $2$-го виду.


\paragraph{Особистий внесок здобувача}

Основні результати, що виносяться на захист, отримані автором
самостійно. Зі статей, опублікованих у співавторстві, до
дисертації включені лише ті результати, що належать автору.


\paragraph{Апробація результатів дисертації}

Основні результати дослідження доповідалися на наукових
конференціях різного рівня та наукових семінарах. Це такі
конференції:
\begin{itemize}
\item Український математичний конгрес, Київ, 21--23 серпня
2001~р.;

\item \ldots
\end{itemize}
Це такі семінари:
\begin{itemize}
\item семінар відділу теорії функцій Інституту математики НАН
України (керівник: чл.-кор. НАН України О.",І.",Степанець);

\item \ldots
\end{itemize}


\paragraph{Публікації}

Основні результати роботи викладено у 6~статтях
% Тут не наводяться всі статті. Це лише приклад
\cite{Bar98fasp1,Bar98fasp2}, опублікованих у виданнях, що внесені
до переліку наукових фахових видань України, та додатково
відображено в матеріалах конференцій~\cite{PrB01umc}.


\paragraph{Зміст роботи}

Тут викладають основні результати дисертації. Це, напевно, зручно
для потенційного читача, для опонентів. Наявність чи відсутність
цього пункту залежить від традицій школи. ВАК не рекомендує і не
забороняє такий пункт у вступі дисертації.

Далі йде безглуздий текст. Не читайте його. Тут немає нічого
розумного (чи хоча б цікавого), оскільки не передбачалося, що
хтось це читатиме. Просто необхідно трохи тексту, щоб сторінку
чимось заповнити. Вважайте, що це щось на кшталт \emph{Lorem
ipsum}. Крім того, за допомогою цієї сторінки з безглуздим текстом
можна порахувати кількість рядків на сторінці та символів у рядку.

Здається, у мене закінчуються запаси безглуздого тексту. Хто б міг
подумати, що так складно писати текст лише для заповнення
сторінки! Він, крім того, ще й неефективний, оскільки не всі букви
можна тут побачити. Але деякі гарні букви і цифри можна
роздивитися: а, б, в,~\ldots, 1, 2, 3,~\ldots, а ще такі: \emph{а,
б, в,~\ldots}.

Досить! Далі йде осмислений (я сподіваюся) текст. Він наведений
тут зовсім не для того, щоб порушити  права Юрія Андруховича чи
видавництва <<Фоліо>>. Просто у мене під руками був електронний
варіант <<Таємниці>>. Чому Рябчук був абсолютним ґуру для них
усіх?

<<По-перше, він у всьому був жахливо переконливий, у всьому "---
як у своїх статтях, так і в розмовах. З ним було безнадійно
полемізувати, його слід було тільки слухати. Йому було на той час
29 років, тобто він був ще й фізично старший від решти товариства,
а це в тому віці суттєво, ця різниця між 29 і, скажімо, 22. Це не
те що 45 і 38 "--- там уже фактично жодної різниці немає. А між 90
і 83 "--- й поготів. Так от, він був старший і досвідченіший, з
ним можна було про все на світі радитися, бо він на той час уже
змінив з десяток різних занять, був тричі одружений і розлучений,
жив самотнім даосом у запущеній старій віллі на Майорівці, з таким
же запущеним старим садом і не менш запущеними старими сусідами.
Він був цілком прозорий від аскетизму (інший тут сказав би, що він
\emph{аж світився}), щодня стояв на голові і правильно, згідно з
Ученням, дихав, а харчувався виключно пісним рисом без солі. На
той час його статті про літературу вже почали публікувати і він
потроху ставав авторитетом не тільки в андеґраунді. Ага, м'ятний
чай "--- він пив багато м'ятного чаю. Його двокімнатне помешкання
у тій віллі являло собою досить інтенсивну суміш з усяких
речей-уламків, але передусім воно було захаращене книгами,
газетами і рукописами. Книги починалися від порогу і ніде не
закінчувалися. Він тримав їх навіть у холодильнику. Якби не книги,
то він і не знав би, на біса йому той холодильник здався. Кажуть,
наче там-таки, у холодильнику, він тримав пришпиленим до задньої
стінки вирізаний з газети портрет Брежнєва. Це називалося
\emph{малим Сибіром}. Ще пару років тому його помешкання стало
такою собі міні-комуною, притулком для тодішніх нефорів: Морозов,
Лишега, Чемодан, Кактус. Останнього я ніколи в житті не бачив, до
сьогодні. Але знаю, що такий був, уявляєш? Здається, саме він
намалював на стіні того сквоту нев'їбенно притягальну фреску з
усіма згаданими особами "--- вони сидять, розпатлані й неголені,
як апостоли, а на столі в них червоне вино і рибина. Не пам'ятаю,
чи мали німби, але припускаю, що цілком могли мати. Це житло стало
такою собі рукавичкою. Кожен із них зносив до Рябчукової хати
всякий непотріб "--- в залежності від того, чим на ту хвилину
перебивався і що звідки вдавалося потягти. Пам'ятаю табличку
ЕКСПОНАТ НА РЕСТАВРАЦІЇ. Ще пам'ятаю ПАЛАТА ДЛЯ НЕДОНОШЕНИХ "--- з
усього випливало, що свого часу котрийсь із гостей цього притулку
підзаробляв на життя пологами. Тепер тобі зрозуміло, чому Микола
Рябчук був абсолютним ґуру?>>


\paragraph{Подяка}

Якщо хочете комусь подякувати, пишіть тут.
%       Вступ
%%
%% This is file `xampl-ch1.tex',
%% generated with the docstrip utility.
%%
%% The original source files were:
%%
%% vakthesis.dtx  (with options: `xampl-ch1')
%% 
%% IMPORTANT NOTICE:
%% 
%% For the copyright see the source file.
%% 
%% Any modified versions of this file must be renamed
%% with new filenames distinct from xampl-ch1.tex.
%% 
%% For distribution of the original source see the terms
%% for copying and modification in the file vakthesis.dtx.
%% 
%% This generated file may be distributed as long as the
%% original source files, as listed above, are part of the
%% same distribution. (The sources need not necessarily be
%% in the same archive or directory.)
%% xampl-ch1.tex  Приклад розділу дисертації
% Приклад назви розділу і мітки, на яку можна посилатися в тексті
\chapter{Подання дійсних чисел рядами~Остроградського $1$-го виду}
\label{ch:o1series}

Це не є справжній розділ дисертації. Це лише приклад, який повинен
допомогти користувачу підготувати свій файл. Але я зробив його з
розділу~1 своєї дисертації.

У цьому розділі вивчається розвинення дійсного числа у
знакозмінний ряд спеціального вигляду, який називається рядом
Остроградського $1$-го виду.

Досліджуються тополого"=метричні та фрактальні властивості множини
неповних сум заданого ряду Остроградського $1$-го виду, а також
властивості розподілів ймовірностей на множині неповних сум.


% Приклад назви підрозділу
\section{Означення ряду Остроградського $1$-го виду}

% Приклад означення
\begin{definition}
% Приклад виноски (\footnote)
\emph{Рядом Остроградського $1$-го виду}\footnote{Далі часто
будемо називати просто \emph{рядом Остроградського}, оскільки ми
не досліджуємо ряди Остроградського $2$-го виду.} називається
скінченний або нескінченний вираз вигляду
% Приклад формули з номером
\begin{equation}\label{eq:o1series}
\frac1{q_1}-\frac1{q_1q_2}+\dots +\frac{(-1)^{n-1}}{q_1q_2\dots
q_n}+\dotsb,
\end{equation}
де $q_n$ "--- натуральні числа і $q_{n + 1}>q_n$ для будь-якого
$n\in\N$. Числа $q_n$ називаються \emph{елементами ряду
Остроградського $1$-го виду}.
\end{definition}

Число елементів може бути як скінченним, так і нескінченним.
У~першому випадку будемо записувати ряд Остроградського у вигляді
% Приклад формули без номера
\[
\frac1{q_1}-\frac1{q_1q_2}+\dots +\frac{(-1)^{n-1}}{q_1q_2\dots
q_n}
\]
або скорочено
\begin{equation*}
\Osign1(q_1,q_2,\dots,q_n)
\end{equation*}
і називати скінченним рядом Остроградського або
$n$\nobreakdash-\hspace{0pt}елементним рядом Остроградського; а в
другому випадку будемо записувати ряд Остроградського у вигляді
% Приклад посилання на формулу
\eqref{eq:o1series} або скорочено
\begin{equation*}
\Osign1(q_1,q_2,\dots,q_n,\dots)
\end{equation*}
і називати нескінченним рядом Остроградського.


\section{Означення та властивості підхідних чисел}

\begin{definition}\label{def:convergent}
\emph{Підхідним числом порядку $k$} ряду Остроградського $1$-го
виду називається раціональне число
\[
\frac{A_k}{B_k} = \frac{1}{q_1}-\frac{1}{q_1q_2}+\dots
+\frac{(-1)^{k-1}}{q_1q_2\dots q_k} = \Osign1(q_1,q_2,\dots,q_k).
\]
\end{definition}

Зрозуміло, що $n$-елементний ряд Остроградського має $n$ підхідних
чисел, причому підхідне число $n$-го порядку $\frac{A_n}{B_n}$
збігається зі значенням цього ряду Остроградського.

% Приклад теореми
\begin{theorem}\label{th:convergents}
Для будь-якого натурального $k$ правильні формули
\begin{equation}\label{eq:convergents}
\left\{
\begin{aligned}
&A_k=A_{k-1}q_k+(-1)^{k-1},\\
&B_k=B_{k-1}q_k=q_1q_2\dots q_k
\end{aligned}
\right.
\end{equation}
\textup(якщо покласти, що $A_0=0$, $B_0=1$\textup).
\end{theorem}

% Приклад доведення
\begin{proof}
Проведемо доведення методом математичної індукції по~$k$. Для
$k=1$ формули правильні. Справді,
\[
\frac{A_1}{B_1}=\frac{1}{q_1}=\frac{A_0q_1+(-1)^0}{B_0q_1}.
\]

Припустимо, що формули \eqref{eq:convergents} правильні для
деякого $k=m$, тобто
\begin{align*}
\left\{
\begin{aligned}
&A_m=A_{m-1}q_m+(-1)^{m-1},\\
&B_m=B_{m-1}q_m=q_1q_2\ldots q_m,
\end{aligned}
\right.
\end{align*}
і доведемо ці формули для $k=m+1$. Маємо
% Приклад формули, що займає більше одного рядка.
% Оточення align, рядки вирівняні по знаку =
\begin{align*}
\frac{A_{m+1}}{B_{m+1}}&=\frac{1}{q_1}-\frac{1}{q_1 q_2}+\dots
+\frac{(-1)^{m-1}}{q_1q_2\dots q_m}+\frac{(-1)^m}{q_1q_2\dots
q_mq_{m + 1}}=\\ &=\frac{A_m}{B_m}+\frac{(-1)^m}{B_mq_{m + 1}} =
\frac{A_mq_{m+1}+(-1)^m}{B_mq_{m + 1}}.
\end{align*}

Отже, за принципом математичної індукції формули
\eqref{eq:convergents} правильні для будь"=якого натурального $k$.
\end{proof}

% Приклад леми
\begin{lemma}
Для будь-якого натурального $k$ правильна рівність
\begin{equation}\label{eq:convergents1}
\frac{A_{k-1}}{B_{k-1}}-\frac{A_k}{B_k}=\frac{(-1)^k}{B_k}.
\end{equation}
\end{lemma}

\begin{lemma}
Для будь-якого натурального $k\geq2$ правильна рівність
\begin{equation}\label{eq:convergents2}
\frac{A_{k-2}}{B_{k-2}}-\frac{A_k}{B_k}=\frac{(-1)^{k-1}(q_k-1)}{B_k}.
\end{equation}
\end{lemma}


\section{Розклад числа у знакозмінний ряд за $1$-м алгоритмом Остроградського}

Почнемо з геометричної ілюстрації алгоритму. Нехай маємо відрізки
$A$ та $B$, $A<B$. Щоб застосувати $1$-й алгоритм Остроградського
до числа $\frac{A}{B}$, будемо відкладати відрізок $A$ на відрізку
$B$, поки не отримаємо залишок $A_1<A$ (див.
% Приклад посилання на малюнок
рис.~\ref{fig:o1alg}). Нехай відрізок $A$ вміщується $q_1$ разів у
відрізку $B$, тоді
\[
B=q_1A+A_1.
\]
Далі відкладемо відрізок $A_1$ не на меншому відрізку $A$ (як у
алгоритмі Евкліда), а на тому ж відрізку $B$ до отримання залишку
$A_2<A_1$. Нехай відрізок $A_1$ вміщується $q_2$ разів у відрізку
$B$, тоді
\[
B=q_2A_1+A_2.
\]
Відкладаючи відрізок $A_2$ знову на відрізку $B$ і~т.~д. до
нескінченності або до отримання нульового залишку, будемо мати
\begin{align*}
&B=q_3A_2+A_3,\\
&B=q_4A_3+A_4
\end{align*}
і~т.~д. З отриманих рівностей випливає, що має місце розклад
\[
\frac AB = \frac{1}{q_1} - \frac{1}{q_1q_2} + \frac{1}{q_1q_2q_3}
- \frac{1}{q_1q_2q_3q_4} + \dotsb,
\]
і тут, як легко бачити,
\[
q_1<q_2<q_3<q_4<\dotsb.
\]

% Приклад малюнка
\begin{figure}[htbp]
\setlength{\unitlength}{1mm}
\begin{center}
\begin{picture}(105,35)
\put(6,28){$A$}\put(51,28){$A_1$} \put(0,25){\line(1,0){58}}
\multiput(0,24)(16,0){4}{\line(0,1){2}}\put(58,24){\line(0,1){2}}
\put(28,19){$B$} \put(75,25){$B=3A+A_1$}
\put(3,9){$A_1$}\put(52,9){$A_2$} \put(0,6){\line(1,0){58}}
\multiput(0,5)(10,0){6}{\line(0,1){2}}\put(58,5){\line(0,1){2}}
\put(28,0){$B$} \put(75,6){$B=5A_1+A_2$}
\end{picture}
\caption{Геометрична ілюстрація $1$-го алгоритму Остроградського:
тут відрізок $A$ вміщується 3~рази у відрізку $B$, відрізок $A_1$
вміщується 5~разів у відрізку $B$ і~т.~д.}
\label{fig:o1alg}
\end{center}
\end{figure}

Таким чином, \emph{$1$-й алгоритм Остроградського} розкладу
дійсного числа $x\in(0,1)$ у знакозмінний ряд полягає в
наступному.

\begin{description}
\item[Крок~1.] Покласти $\alpha_0=x$, $i=1$.

\item[Крок~2.] Знайти такі числа $q_i$ та $\alpha_i$, що
\[
1=q_i\alpha_{i-1}+\alpha_i \quad \text{і} \quad
0\leq\alpha_i<\alpha_{i-1}.
\]

\item[Крок~3.] Якщо $\alpha_i=0$, то припинити обчислення. Інакше
"--- збільшити $i$ на $1$ та перейти до кроку~2.
\end{description}

\begin{theorem}\label{thm:ostrogradsky}
Кожне дійсне число $x\in(0,1)$ можна подати у вигляді ряду
Остроградського $1$-го виду~\eqref{eq:o1series}. Причому, якщо
число $x$ ірраціональне, то це можна зробити єдиним чином і
вираз~\eqref{eq:o1series} має при цьому нескінченне число
доданків; якщо ж число $x$ раціональне, то його можна подати у
вигляді~\eqref{eq:o1series} зі скінченним числом доданків двома
різними способами:
\[
x=\Osign1(q_1,q_2,\dots,q_{n-1},q_n)=\Osign1(q_1,q_2,\dots,q_{n-1},q_n-1,q_n).
\]
\end{theorem}

% Приклад посилання на таблицю
У табл.~\ref{tab:ellipse.hyperbola.parabola} наведені деякі
формули для еліпса, гіперболи і параболи.

% Приклад таблиці
\begin{table}[htbp]
\caption{Еліпс, гіпербола і парабола. Деякі формули}
\label{tab:ellipse.hyperbola.parabola}
\begin{tabularx}{\textwidth}{|X|c|c|c|}
\hline
                   & Еліпс                                    & Гіпербола                                & Парабола          \\
\hline
Канонічне рівняння & $\frac{x^2}{a^2}+\frac{y^2}{b^2}=1$      & $\frac{x^2}{a^2}-\frac{y^2}{b^2}=1$      & $y^2=2px$         \\
Ексцентриситет     & $\varepsilon=\sqrt{1-\frac{b^2}{a^2}}<1$ & $\varepsilon=\sqrt{1+\frac{b^2}{a^2}}>1$ & $\varepsilon=1$   \\
Фокуси             & $(a\varepsilon,0)$, $(-a\varepsilon,0)$  & $(a\varepsilon,0)$, $(-a\varepsilon,0)$  & $(\frac{p}{2},0)$ \\
\hline
\multicolumn{4}{|l|}{Корн~Г., Корн~Т. Справочник по математике. М., 1974. С.~72.} \\
\hline
\end{tabularx}
\end{table}


\section{Множина неповних сум ряду Остроградського та розподіли ймовірностей на ній}

Візьмемо довільну \emph{фіксовану} послідовність $\{q_k\}$
натуральних чисел з умовою $q_{k+1}>q_k$ для всіх $k\in\N$ і
розглянемо їй відповідний ряд Остроградського $1$-го
виду~\eqref{eq:o1series} з сумою $r$. Число $r$ можна записати у
вигляді
\begin{equation}
r=d-b, \quad \text{де} \quad d=\sum_{i=1}^\infty
\frac1{q_1q_2\dots q_{2i-1}}, \quad b=\sum_{i=1}^\infty
\frac1{q_1q_2\dots q_{2i}}.
\end{equation}

% Приклад назви пункту
\subsection{Тополого-метричні та фрактальні властивості множини
неповних сум ряду Остроградського}

\emph{Циліндром} рангу $m$ з основою $c_1c_2\dots c_m$ називається
множина $\Delta'_{c_1c_2\dots c_m}$ всіх неповних сум, які мають
зображення $\Delta_{c_1c_2\dots c_ma_{m+1}\dots a_{m+k}\dots}$, де
$a_{m+j}\in\set{0,1}$ для будь-якого $j\in\N$. Очевидно, що
\[
\Delta'_{c_1c_2\dots c_ma}\subset\Delta'_{c_1c_2\dots c_m}, \quad
a\in\set{0,1}.
\]

% Приклад теоремоподібної структури з додатковою інформацією в заголовку
\begin{definition}[{\cite[с.~59]{Pra98}}]
\emph{Фракталом} називається кожна континуальна обмежена множина
простору $\R^1$, яка має тривіальну (рівну $0$ або $\infty$)
$H_\alpha$-міру Хаусдорфа, порядок $\alpha$ якої дорівнює
топологічній розмірності.
\end{definition}

Ті нуль-множини Лебега простору $\R^1$, розмірність
Хаусдорфа\nobreakdash--\hspace{0pt}Безиковича яких дорівнює $1$,
називаються \emph{суперфракталами}, а континуальні множини, що
мають нульову розмірність Хаусдорфа--Безиковича, називаються
\emph{аномально фрактальними}.


% Приклад висновків до розділу
\section*{Висновки до розділу~\ref{ch:o1series}}

У розділі~\ref{ch:o1series} введене поняття ряду Остроградського
$1$-го виду та його підхідних чисел, запропоновані деякі
властивості підхідних чисел. Доведено, що кожне дійсне число
$x\in(0,1)$ можна подати у вигляді ряду Остроградського $1$-го
виду: ірраціональне "--- єдиним чином у вигляді нескінченного ряду
Остроградського, раціональне "--- двома різними способами у
вигляді скінченного ряду Остроградського. Ці результати не є
новими, їх можна знайти, наприклад, у
роботах~\cite{Rem51,Sie11STNW,Pie29,VaZ75,Sha86} та~ін. Вони
наведені тут для повноти викладу.

Новими в цьому розділі є результати, що стосуються неповних сум
ряду Остроградського. Описані тополого-метричні та фрактальні
властивості множини неповних сум ряду Остроградського. Описано
множини чисел, ряди Остроградського яких є простими і густими
відповідно. Доведено, що випадкова неповна сума ряду
Остроградського має або дискретний розподіл або сингулярний
розподіл канторівського типу. Досліджено поведінку на
нескінченності модуля характеристичної функції випадкової неповної
суми ряду Остроградського.
%         Розділ 1
%\include{xampl-ch2}%         Розділ 2
%\include{xampl-ch3}%         Розділ 3
%\include{xampl-ch4}%         Розділ 4 і т. д. ще скільки потрібно розділів
%%
%% This is file `xampl-concl.tex',
%% generated with the docstrip utility.
%%
%% The original source files were:
%%
%% vakthesis.dtx  (with options: `xampl-concl')
%% 
%% IMPORTANT NOTICE:
%% 
%% For the copyright see the source file.
%% 
%% Any modified versions of this file must be renamed
%% with new filenames distinct from xampl-concl.tex.
%% 
%% For distribution of the original source see the terms
%% for copying and modification in the file vakthesis.dtx.
%% 
%% This generated file may be distributed as long as the
%% original source files, as listed above, are part of the
%% same distribution. (The sources need not necessarily be
%% in the same archive or directory.)
%% xampl-concl.tex  Приклад висновків до дисертації
\chapter*{Висновки}

Це не є справжні висновки до дисертації. Це лише приклад, який
повинен допомогти користувачу підготувати свій файл. Але я зробив
його з висновків до своєї дисертації.

Ряди Остроградського $1$-го виду дозволяють розширити можливості
формального задання і аналітичного дослідження фрактальних множин,
сингулярних мір, недиференційовних функцій та інших об'єктів зі
складною локальною будовою.

В дисертаційній роботі отримано такі результати.
\begin{itemize}
\item Розроблено основи метричної теорії чисел, представлених
рядами Остроградського $1$-го виду. Зокрема, досліджено геометрію
розвинень чисел в ряди Остроградського $1$-го виду, отримано
основне метричне відношення та його оцінки, які допомагають у
розв'язанні задач про міру Лебега множин чисел з умовами на
елементи зображення.

\item Знайдено умови нуль-мірності (додатності міри) певних класів
замкнених ніде не щільних множин чисел, заданих умовами на
елементи їх розвинення в ряд Остроградського $1$-го виду.

\item Вивчено тополого-метричні та фрактальні властивості множини
неповних сум заданого ряду Остроградського $1$-го виду та
розподілів ймовірностей на ній.

\item Досліджено структуру та властивості випадкової величини з
незалежними різницями послідовних елементів її представлення рядом
Остроградського $1$-го виду.

\item Вивчено диференціальні та фрактальні властивості однієї
функції, заданої перетворювачем елементів ряду Остроградського
$1$-го виду її аргумента в двійкові цифри значення функції.
\end{itemize}

Як виявилося, існують принципові відмінності метричної теорії
рядів Остроградського та метричної теорії ланцюгових дробів.
Зокрема, існує клас замкнених ніде не щільних множин додатної міри
Лебега, описаних в термінах елементів ряду Остроградського. В той
же час, аналогічні множини, задані у термінах елементів
ланцюгового дробу, мають нульову міру Лебега.

Проведені дослідження лежать в руслі сучасних математичних
досліджень об'єктів зі складною локальною поведінкою (будовою),
пов'язаних з ланцюговими дробами, рядами Люрота,
$\beta$-розкладами тощо, інтерес до яких у світі достатньо
високий. Отримані результати та запропоновані методи можуть бути
корисними при розв'язанні задач метричної теорії чисел,
представлених рядами Остроградського $2$-го виду або іншими
зображеннями з нескінченним алфавітом.
%       Висновки
%%
%% This is file `xampl-bib.tex',
%% generated with the docstrip utility.
%%
%% The original source files were:
%%
%% vakthesis.dtx  (with options: `xampl-bib')
%% 
%% IMPORTANT NOTICE:
%% 
%% For the copyright see the source file.
%% 
%% Any modified versions of this file must be renamed
%% with new filenames distinct from xampl-bib.tex.
%% 
%% For distribution of the original source see the terms
%% for copying and modification in the file vakthesis.dtx.
%% 
%% This generated file may be distributed as long as the
%% original source files, as listed above, are part of the
%% same distribution. (The sources need not necessarily be
%% in the same archive or directory.)
%% xampl-bib.tex  Приклад файла-оболонки для списку/списків літератури
% Рядки, що починаються з %GATHER, призначені для WinEdt
% Якщо є лише один список літератури, оточення bibset не потрібно використовувати
%GATHER{xampl-thesis.bib}
\begin{bibset}{Список використаних джерел}
\bibliographystyle{gost2008}
% Для сортування літератури за алфавітом використовуйте
%\bibliographystyle{gost2008s}
\bibliography{xampl-thesis}
\end{bibset}
%GATHER{xampl-mybib.bib}
\begin{bibset}[a]{Список публікацій автора}
\bibliographystyle{gost2008}
\bibliography{xampl-mybib}
\end{bibset}
%         Список використаних джерел (+список публікацій автора)
%\appendix
%\include{xampl-app1}%        Додаток 1
%\include{xampl-app2}%        Додаток 2 і т. д. ще скільки потрібно додатків

\end{document}
