%% mon2017dev-bib.tex  Приклад списку використаних джерел (для mon2017dev.tex)

% TODO: Це потрібно для статті Sie11STNW з файла xampl-thesis.bib.
% Ці команди жорстко прописані у полі note.
% Треба зробити зразок без цього.
\def\selectlanguageifdefined#1{
\expandafter\ifx\csname date#1\endcsname\relax
\else\selectlanguage{#1}\fi}
\providecommand*{\BibEmph}[1]{#1}

% Рекомендований перелік стилів оформлення списку наукових публікацій
% Додаток 3 до Вимог до оформлення дисертації (пункт 11 розділу ІІІ)
% https://zakon.rada.gov.ua/go/z0155-17

% ------------------------------------------------------------------------------
% Стиль оформлення		BibTeX-стиль		Джерело
% списку наукових публікацій
% ------------------------------------------------------------------------------
% ДСТУ 8302:2015		dstu2015*		https://codeberg.org/mdmisch/dstu
% 				gost2008*		https://ctan.org/pkg/gost
% MLA (Modern Language		mla			https://ctan.org/pkg/mla
%   Association)
% APA (American Psychological	apalike			https://ctan.org/pkg/bibtex
%   Association)[1][2]		apacite			https://ctan.org/pkg/apacite
% Chicago/Turabian[1]		achicago		https://ctan.org/pkg/achicago-bst
% Harvard[1]			[3]			https://ctan.org/pkg/harvard
% ACS (American			achemso/biochem		https://ctan.org/pkg/achemso
%   Chemical Society)
% AIP (American			aipauth4-2/aipnum4-2	https://ctan.org/pkg/revtex
%   Institute of Physics)
% IEEE (Institute of Electrical	ieeetr			https://ctan.org/pkg/bibtex
%   and Electronics Engineers)	IEEEtran*		https://ctan.org/pkg/ieeetran
% Vancouver[1]			[3]			https://ctan.org/pkg/vancouver
% OSCOLA			???
% APS (American			apsrev4-2/apsrmp4-2	https://ctan.org/pkg/revtex
%   Physical Society)[1]
% Springer MathPhys[1]		spmpsci			https://resource-cms.springernature.com/springer-cms/rest/v1/content/25980412/data/v2
% ------------------------------------------------------------------------------

% [1] Оригінальне гіперпосилання в додатку 3 вже недоступне.
% Інструкції від Springer щодо підготовки рукописів
% (зокрема щодо оформлення списку посилань) тепер на цій сторінці:
% https://www.springer.com/gp/authors-editors/book-authors-editors/your-publication-journey/manuscript-preparation
% [2] Оригінальне гіперпосилання в додатку 3 вже переадресовує на іншу сторінку.
% Інструкції від Elsevier щодо оформлення списку посилань
% у журналі «Learning and Instruction» тепер доступні тут:
% https://www.sciencedirect.com/journal/learning-and-instruction/publish/guide-for-authors#68000
% [3] Harvard і Vancouver — це радше системи оформлення посилань
% (у текстовій і числовій формі відповідно),
% ніж конкретні стилі оформлення списку наукових публікацій.
% Але на CTAN є пакунки з такими іменами, які містять кілька BibTeX-стилів.
% Можливо, вони будуть корисні у зв'язку з рекомендаціями додатку 3.

% Корисний огляд різноманітних BibTeX-стилів від Reed College:
% https://www.reed.edu/it/help/LaTeX/bibtexstyles.html

\begin{bibset}{Список використаних джерел}
  \bibliographystyle{ieeetr}% або інший BibTeX-стиль, наприклад, gost2008s
  %
  % Якщо не треба нумерація з крапкою, можна закоментувати наступні три рядки.
  \makeatletter
  \renewcommand\@biblabel[1]{#1.}
  \makeatother
  \bibliography{xampl-thesis,xampl-mybib}
\end{bibset}
