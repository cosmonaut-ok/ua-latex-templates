%% mon2017dev-intro.tex  Приклад вступу до дисертації (для mon2017dev.tex)

\chapter*{Вступ}

\paragraph{Обґрунтування вибору теми дослідження}

Висвітлюється зв'язок теми дисертації
із сучасними дослідженнями у відповідній галузі знань
шляхом критичного аналізу
з визначенням сутності наукової проблеми або завдання.

% За наявності у вступі можуть також вказуватися
\paragraph{Зв'язок роботи з науковими програмами, планами, темами, грантами}

Вказується, в рамках яких програм, тематичних планів, наукових тематик і грантів,
зокрема галузевих, державних та/або міжнародних,
виконувалося дисертаційне дослідження,
із зазначенням номерів державної реєстрації науково-дослідних робіт
і найменуванням організації, де виконувалася робота.

\paragraph{Мета і завдання дослідження}

Відповідно до предмета та об'єкта дослідження.

\paragraph{Методи дослідження}

Перераховуються використані наукові методи дослідження
та змістовно відзначається, що саме досліджувалось кожним методом;
обґрунтовується вибір методів,
що забезпечують достовірність отриманих результатів та висновків.

\paragraph{Наукова новизна отриманих результатів}

Аргументовано, коротко та чітко представляються
основні наукові положення, які виносяться на захист,
із зазначенням відмінності одержаних результатів від відомих раніше.

% За наявності у вступі можуть також вказуватися
\paragraph{Практичне значення отриманих результатів}

Надаються відомості про використання результатів досліджень
або рекомендації щодо їх практичного використання.

\paragraph{Особистий внесок здобувача}

Якщо у дисертації використано ідеї або розробки,
що належать співавторам, разом з якими здобувачем опубліковано наукові праці,
обов'язково зазначається
конкретний особистий внесок здобувача в такі праці або розробки;
здобувач має також додати посилання на дисертації співавторів,
у яких було використано результати спільних робіт.

% Апробація матеріалів дисертації
\begin{approval}
У вступі подається апробація матеріалів дисертації
(зазначаються назви конференції, конгресу, симпозіуму, семінару, школи,
місце та дата проведення).

Основні результати дослідження доповідалися на наукових
конференціях різного рівня та наукових семінарах. Це такі
конференції:
\begin{itemize}
\item Український математичний конгрес, Київ, 21--23~серпня
2001~р. \participation{секційна доповідь};
% Команда \participation працює так,
% що форму участі у вступі не буде показувати,
% але якщо цей текст перенести в додаток, то буде.

\item Звітна конференція викладачів, аспірантів та докторантів університету,
  Київ, 1--2~лютого 2002~р. \participation{пленарна доповідь};

\item Конференція молодих вчених <<Сучасна алгебра і топологія>>,
  Одеса, 15--20~серпня 2003~р. \participation{стендова доповідь};

\item \ldots
\end{itemize}
Це такі семінари:
\begin{itemize}
\item семінар відділу теорії функцій Інституту математики НАН
України (керівник: чл.-кор. НАН України О.",І.",Степанець);
% Говорити про форму участі в семінарі немає сенсу, на мій погляд.
% (Якщо розуміти семінар як такий науковий захід,
% де тільки один доповідач (або небагато), а всі інші "--- слухачі.
% Тоді не може бути ніякої пленарної, секційної чи стендової доповіді,
% як на конференції, де може бути кілька паралельних потоків.
% Можливо, в інших науках семінаром називають щось інше.)
% Але за потреби команду \participation тут теж можна використовувати.

\item \ldots
\end{itemize}
\end{approval}

\paragraph{Структура та обсяг дисертації}

Анонсується структура дисертації, зазначається її загальний обсяг.
