%% mon2017dev.tex  Приклад головного файла дисертації (для mon2017dev.cls)

% Це просто файл-приклад, який дисертантки і дисертанти можуть використати
% як основу для своєї дисертації.
% Але в цьому базовому вигляді він може не підходити для всіх дисертацій.
% Треба наповнити його конкретним змістом,
% а також зробити певні налаштування згідно зі своїми потребами:
% вибрати потрібні опції класу,
% підключити корисні додаткові пакунки,
% налаштувати параметри сторінки (зокрема розміри берегів),
% означити специфічні команди, які використовуються в дисертації, тощо.

\documentclass[
  % type=phd,% тип дисертації
    %%%%%% Ступені згідно з Порядком присудження наукових ступенів (Постанова КМ)
    % c    кандидат <галузь науки> (типово)
    % d    доктор <галузь науки>
    %%%%%% Ступені згідно із Законом України «Про вищу освіту»
    % phd  доктор філософії
    % artd доктор мистецтва
    % scd  доктор наук
  % instnameorder=asc,% порядок розміщення на титульному аркуші
                      % назви установи і назви органу, до сфери управління якого належить установа
    % desc низхідний (типово)
    % asc  висхідний
    % ascd висхідний порядок з відмінюванням
  % guide=mon2017dev.naiau,% модуль для підтримки специфічних вимог
    % karazin Харківський національний університет імені В. Н. Каразіна
    %         -- галузь науки для типу дисертації c і d
    %         -- дрібні зміни в оформленні
    % knu     Київський національний університет імені Тараса Шевченка
    %         -- повтор однакових установ двічі
    % naiau   Національна академія внутрішніх справ
    %         -- лапки, тире точно, як у зразку з наказу МОН
    %         -- дрібні зміни в оформленні
    % --------------------------------------------------------------------------
    % kivi    у стилі Киви
    %         -- дисертація за здоров'я наукового рівня
    %            dissertation to the health of the science level
]{mon2017dev}[2021/04/12]

% Налагодження кодування шрифта, кодування вхідного файла
% та вибір необхідних мов
\usepackage[T2A]{fontenc}
\usepackage[cp1251]{inputenc}
% TODO: Перевірити, якщо головна мова документа -- english.
% Ще не всі службові слова перекладено (тільки ті, що використовуються в анотації).
% Не рекомендовано використовувати для документів,
% у яких головна мова -- не ukrainian (а english чи інша).
\usepackage[russian,english,ukrainian]{babel}

% Підключення необхідних пакетів. Наприклад,
% Пакети AMS для підтримки математики, теорем, спеціальних шрифтів
\usepackage[intlimits]{amsmath}
\allowdisplaybreaks
\usepackage{amsthm}
\usepackage{amssymb}
% Налагодження нумерованих списків
%\usepackage{enumerate}
% Гіпертекстові документи
%\usepackage{hyperref}
% або лише спеціальне форматування URL
%\usepackage{url}
% У списку літератури зворотні вказівки на посилання
%\usepackage{backref}
% Сортування посилань
%\usepackage[noadjust]{cite}
% Останні два пакети несумісні між собою. Крім того, конфліктують з цим класом!
% Таблиці зі стовпчиками, що розтягуються
\usepackage{tabularx}
% Включення факсимільних підписів і взагалі робота з ілюстраціями, кольором тощо
% \usepackage{graphicx}

% Налагодження параметрів сторінки (зокрема берегів).
% Наприклад, за допомогою пакета geometry
% УВАГА!
% Параметри сторінки не зафіксовані жорстко вимогами до оформлення дисертацій
% і можуть за потреби дещо змінюватися.
% Це просто приклад.
% Варто встановити свої параметри залежно від особливостей тексту дисертації
% чи технічних особливостей майстерні, яка робитиме оправу.
\usepackage{geometry}
\geometry{hmargin={30mm,15mm},lines=29,vcentering}

% Означення теорем (теоремоподібних структур)
% Класичний варіант: для кожної теореми свій лічильник,
% тобто теорема 1.1, лема 1.1, теорема 1.2
\theoremstyle{plain}
\newtheorem{theorem}{Теорема}[chapter]
\newtheorem{lemma}{Лема}[chapter]
\newtheorem{corollary}{Наслідок}[chapter]
\theoremstyle{definition}
\newtheorem{definition}{Означення}[chapter]
\newtheorem{example}{Приклад}[chapter]
\theoremstyle{remark}
\newtheorem{remark}{Зауваження}[chapter]
% Цікавий варіант: всі теореми нумеруються одним лічильником,
% тобто теорема 1.1, лема 1.2, теорема 1.3
%\theoremstyle{plain}
%\newtheorem{theorem}{Теорема}[chapter]
%\newtheorem{lemma}[theorem]{Лема}
%\newtheorem{corollary}[theorem]{Наслідок}
%\theoremstyle{definition}
%\newtheorem{definition}[theorem]{Означення}
%\newtheorem{example}[theorem]{Приклад}
%\theoremstyle{remark}
%\newtheorem{remark}[theorem]{Зауваження}

% Локальні означення
\newcommand{\N}{\mathbb{N}}
\newcommand{\Z}{\mathbb{Z}}
\newcommand{\Q}{\mathbb{Q}}
\newcommand{\R}{\mathbb{R}}
\newcommand{\set}[1]{\left\{#1\right\}}
\newcommand{\abs}[1]{\left\lvert#1\right\rvert}
\newcommand{\norm}[2][]{\left\lVert#2\right\rVert_{#1}}
% Це потрібно для скороченого запису ряду Остроградського
\newcommand{\Osign}[1]{\mathrm{O}^{#1}}

% Якщо потрібно працювати лише з деякими розділами
%\includeonly{xampl-ch1,xampl-bib}

% Інформація про використані пакети тощо.
% Може знадобитися для відлагодження класу документа
%\listfiles

\begin{document}

% Назва дисертації
\title(uk){Формування професійної іншомовної компетентності майбутніх філологів
  засобами~мультимедійних технологій}
\title(en){Formation of future philologists' professional foreign competence
  by the means of multimedia}

% Прізвище, ім'я, по батькові здобувача
\author(uk){Сорокіна Наталія Володимирівна}
\author(en){Sorokina Nataliya Volodymyrivna}

% Факсимільний підпис автора у файлі sorokina-sig.pdf, .eps, .jpeg тощо
% (зсув по x, зсув по y)
% [параметри команди \includegraphics]
% \facsimilesig{author}(-60,-12)[width=70pt]{sorokina-sig}

% Прізвище, ім'я, по батькові наукового керівника/консультанта
\supervisor(uk){Редько Валерій Григорович}
% Науковий ступінь, вчене звання наукового керівника/консультанта
               {кандидат педагогічних наук, доцент}
% Установа, де працює науковий керівник/консультант, і посада
               {Інститут педагогіки НАПН України,
                завідувач відділу навчання іноземних мов}
% \supervisor(en){Redko Valeriy Hryhorovych}
%                {candidate of pedagogical sciences, associate professor}
%                {}
% \supervisor(uk){Альфа Бета Гамма}
%                {доктор фізико-математичних наук, професор}
%                {}
% \supervisor(en){Alpha Beta Gamma}
%                {doctor of physical and mathematical sciences, professor}
%                {}

% Спеціальність
% Якщо код спеціальності присутній у CSV-файлі, то клас читає інформацію з нього.
% Потрібний CSV-файл вибирається залежно від типу дисертації та основної мови документа.
% За потреби можна підключити інший CSV-файл з шифрами і назвами спеціальностей
% або у факультативному аргументі команди задати всі дані.
% Для типу дисертації c або d
% можна використовувати обидві команди \specialitysci і \specialityedu одночасно,
% щоб отримати наче «змішаний» ступінь:
% тоді на титульному аркуші й в анотації
% буде написано шифри й назви спеціальностей за обома переліками,
% а для типу дисертації c ще й ступінь у вигляді
% «кандидата ... наук (доктора філософії)»

% Варіант для дисертацій на здобуття наукового ступеня кандидата чи доктора ... наук
% (згідно з Переліком наукових спеціальностей)
% Цю команду треба використовувати для типу дисертації c або d.
\specialitysci(uk)[
  % specialityname=Теорія і методика професійної освіти,% спеціальність
  % degreefield=педагогічні,                            % галузь науки, за якою присуджується науковий ступінь
  % specialityfile=<filename>.csv
]{13.00.04}                                             % шифр спеціальності
\specialitysci(en)[
  % specialityname=Theory and methodology of professional education,
  % degreefield=pedagogical,
  % specialityfile=<filename>.csv
]{13.00.04}

% Варіант для дисертацій на здобуття наукового ступеня доктора філософії, доктора мистецтва чи доктора наук
% (згідно з Переліком галузей знань і спеціальностей, за якими здійснюється підготовка здобувачів вищої освіти
% Цю команду треба використовувати для типу дисертації phd, artd або scd,
% але також можна для типу дисертації c або d разом з командою \specialitysci.
% \specialityedu(uk)[
%   % specialityname=Дошкільна освіта,% найменування спеціальності
%   % fieldcode=01,                   % шифр галузі
%   % fieldname=Освіта/Педагогіка,    % галузь знань
%   % specialityfile=<filename>.csv
% ]{012}                              % код спеціальності
% \specialityedu(en)[
%   % specialityname=Preschool education,
%   % fieldcode=01,
%   % fieldname=Education/Pedagogics,
%   % specialityfile=<filename>.csv
% ]{012}

% Звичайно, доступна команда \speciality, як у попередніх версіях.
% Клас означує її як одну з команд \specialitysci чи \specialityedu
% залежно від типу дисертації.
% Тому можна використовувати її у такій формі, наприклад:
% \speciality(uk)[degreefield={фізико"=математичні}]{01.01.01}
% \speciality(en){01.01.01}

% Індекс за УДК
\udc{371.15:81'243}

% Установа, де виконана робота, і місто
\institution(uk)[
  altname=Інститут педагогіки НАПН України% альтернативна назва установи (пишеться в анотації в описі дисертації)
]{Інститут педагогіки, Національна академія педагогічних наук України}
 {Київ}
\institution(en)[
  altname=Institute of Pedagogics of the NAPS of Ukraine
]{Institute of Pedagogics, National Academy of Pedagogical Sciences of Ukraine}
 {Kyiv}

% Команду \council переписано в стилі команди \institution:
% ключ institution задає «стандартну» назву установи, у спеціалізованій вченій раді якої проводиться захист дисертації
% (з вказуванням назви органу, до сфери управління якого належить заклад, установа),
% а ключ altname — «альтернативну» (тобто скорочену, для анотації).
% Якщо факультативний аргумент відсутній, то клас вважає,
% що захист проводиться в тій самій установі, де здійснювалася підготовка здобувача,
% а отже, немає потреби писати назву цієї установи двічі на титульному і в анотації.
% Але за потреби можна вручну повторити тут назву установи, і буде повтор на титульному і в анотації.
\council(uk)[
  institution={Національний педагогічний університет імені~М.~П.~Драгоманова,
    Міністерство освіти і науки України},
  altname=Національний педагогічний університет імені~М.~П.~Драгоманова,
  address={01601 м.~Київ, вул.~Пирогова, 9},
  % town=Київ
]{Д~26.053.01}
\council(en)[
  institution={National Pedagogical Dragomanov University,
    Ministry of Education and Science of Ukraine},
  altname=National Pedagogical Dragomanov University,
  % town=Kyiv
]{D~26.053.01}

% Рік, коли написана дисертація
\date{2016}

% \secret{Таємно}

% Тут буде титульний аркуш
\maketitle

% Анотація
%% mon2017dev-abs.tex  Приклад анотації (для mon2017dev.tex)

\begin{abstract}[
  language=ukrainian,% мова анотації
  % chapter=Реферат, % заголовок розділу або false, щоб не робити заголовок (типово Анотація/Abstract)
  % header=false     % автоматична генерація опису дисертації (типово true)
]
  У дисертації розглянуто проблему
  формування професійної іншомовної компетентності майбутніх філологів
  засобами мультимедійних технологій.
  У дисертації розглянуто проблему
  формування професійної іншомовної компетентності майбутніх філологів
  засобами мультимедійних технологій і ще кілька слів.

  Другий абзац $x + y = z^2$.

  \emph{Третій абзац.}
  У дисертації розглянуто проблему
  формування професійної іншомовної компетентності майбутніх філологів
  засобами мультимедійних технологій.
  У дисертації розглянуто проблему
  формування професійної іншомовної компетентності майбутніх філологів
  засобами мультимедійних технологій і ще кілька слів.
  У дисертації розглянуто проблему
  формування професійної іншомовної компетентності майбутніх філологів
  засобами мультимедійних технологій.

  \keywords{%
    професійна іншомовна компетентність,
    майбутній філолог,
    мультимедійна навчальна презентація,
    педагогічна технологія%
  }
\end{abstract}

\begin{abstract}[
  language=english,
  % chapter=Summary,
  % header=false
]
  In the thesis, we consider a problem
  of formation of future philologist's professional foreign competence
  by means of multimedia.
  In the thesis, we consider a problem
  of formation of future philologist's professional foreign competence
  by means of multimedia.

  Second paragraph $x + y = z^2$.

  \emph{Third paragraph.}
  In the thesis, we consider a problem
  of formation of future philologist's professional foreign competence
  by means of multimedia.
  In the thesis, we consider a problem
  of formation of future philologist's professional foreign competence
  by means of multimedia.
  In the thesis, we consider a problem
  of formation of future philologist's professional foreign competence
  by means of multimedia.
  In the thesis, we consider a problem
  of formation of future philologist's professional foreign competence
  by means of multimedia.
  In the thesis, we consider a problem
  of formation of future philologist's professional foreign competence
  by means of multimedia.

  \keywords{%
    professional foreign competence,
    future filologist,
    multimedia learning presentation,
    pedagogical technology%
  }
\end{abstract}

\nocite{Bar98fasp1,Bar98fasp2,PrB01umc}

% Рекомендований перелік стилів оформлення списку наукових публікацій
% Додаток 3 до Вимог до оформлення дисертації (пункт 11 розділу ІІІ)
% https://zakon.rada.gov.ua/go/z0155-17

% ------------------------------------------------------------------------------
% Стиль оформлення		BibTeX-стиль		Джерело
% списку наукових публікацій
% ------------------------------------------------------------------------------
% ДСТУ 8302:2015		dstu2015*		https://codeberg.org/mdmisch/dstu
% 				gost2008*		https://ctan.org/pkg/gost
% MLA (Modern Language		mla			https://ctan.org/pkg/mla
%   Association)
% APA (American Psychological	apalike			https://ctan.org/pkg/bibtex
%   Association)[1][2]		apacite			https://ctan.org/pkg/apacite
% Chicago/Turabian[1]		achicago		https://ctan.org/pkg/achicago-bst
% Harvard[1]			[3]			https://ctan.org/pkg/harvard
% ACS (American			achemso/biochem		https://ctan.org/pkg/achemso
%   Chemical Society)
% AIP (American			aipauth4-2/aipnum4-2	https://ctan.org/pkg/revtex
%   Institute of Physics)
% IEEE (Institute of Electrical	ieeetr			https://ctan.org/pkg/bibtex
%   and Electronics Engineers)	IEEEtran*		https://ctan.org/pkg/ieeetran
% Vancouver[1]			[3]			https://ctan.org/pkg/vancouver
% OSCOLA			???
% APS (American			apsrev4-2/apsrmp4-2	https://ctan.org/pkg/revtex
%   Physical Society)[1]
% Springer MathPhys[1]		spmpsci			https://resource-cms.springernature.com/springer-cms/rest/v1/content/25980412/data/v2
% ------------------------------------------------------------------------------

% [1] Оригінальне гіперпосилання в додатку 3 вже недоступне.
% Інструкції від Springer щодо підготовки рукописів
% (зокрема щодо оформлення списку посилань) тепер на цій сторінці:
% https://www.springer.com/gp/authors-editors/book-authors-editors/your-publication-journey/manuscript-preparation
% [2] Оригінальне гіперпосилання в додатку 3 вже переадресовує на іншу сторінку.
% Інструкції від Elsevier щодо оформлення списку посилань
% у журналі «Learning and Instruction» тепер доступні тут:
% https://www.sciencedirect.com/journal/learning-and-instruction/publish/guide-for-authors#68000
% [3] Harvard і Vancouver — це радше системи оформлення посилань
% (у текстовій і числовій формі відповідно),
% ніж конкретні стилі оформлення списку наукових публікацій.
% Але на CTAN є пакунки з такими іменами, які містять кілька BibTeX-стилів.
% Можливо, вони будуть корисні у зв'язку з рекомендаціями додатку 3.

% Корисний огляд різноманітних BibTeX-стилів від Reed College:
% https://www.reed.edu/it/help/LaTeX/bibtexstyles.html

\begin{bibset}% [a]
  {Список публікацій здобувача за~темою~дисертації}
  % {Список публікацій здобувача}
  \bibliographystyle{ieeetr}% або інший BibTeX-стиль, наприклад, gost2008
  %
  % Якщо не треба нумерація з крапкою, можна закоментувати наступні три рядки.
  \makeatletter
  \renewcommand\@biblabel[1]{#1.}
  \makeatother
  \bibliography{xampl-mybib}
\end{bibset}

% УВАГА!

% Автоматичне копіювання списку публікацій здобувача з анотації в додаток
% працює тільки тоді, коли цей список генерується засобами BibTeX з бази даних.
% Тоді BibTeX створює спеціальний .bbl-файл,
% який LaTeX спочатку включає до анотації, а потім — до додатка.

% Якщо список публікацій здобувача створено вручну
% (у вигляді стандартного оточення thebibliography),
% його треба винести в окремий .bbl-файл зі спеціальним іменем.
% Якщо mon2017dev.tex — головний файл дисертації,
% то цей файл має називатися mon2017dev1.bbl.
% Тоді автоматичне копіювання теж працюватиме.

% У файлі анотації (mon2017dev-abs.tex)
% достатньо залишити оточення bibset у такому вигляді, як у зразку.
% У файлі додатку (mon2017dev-app1.tex)
% достатньо залишити команду \repeatauthorpublications.
% Тобто нічого не змінювати в цих файлах.


% Зміст
\tableofcontents

% Розділи дисертації в окремих файлах
%% mon2017dev-intro.tex  Приклад вступу до дисертації (для mon2017dev.tex)

\chapter*{Вступ}

\paragraph{Обґрунтування вибору теми дослідження}

Висвітлюється зв'язок теми дисертації
із сучасними дослідженнями у відповідній галузі знань
шляхом критичного аналізу
з визначенням сутності наукової проблеми або завдання.

% За наявності у вступі можуть також вказуватися
\paragraph{Зв'язок роботи з науковими програмами, планами, темами, грантами}

Вказується, в рамках яких програм, тематичних планів, наукових тематик і грантів,
зокрема галузевих, державних та/або міжнародних,
виконувалося дисертаційне дослідження,
із зазначенням номерів державної реєстрації науково-дослідних робіт
і найменуванням організації, де виконувалася робота.

\paragraph{Мета і завдання дослідження}

Відповідно до предмета та об'єкта дослідження.

\paragraph{Методи дослідження}

Перераховуються використані наукові методи дослідження
та змістовно відзначається, що саме досліджувалось кожним методом;
обґрунтовується вибір методів,
що забезпечують достовірність отриманих результатів та висновків.

\paragraph{Наукова новизна отриманих результатів}

Аргументовано, коротко та чітко представляються
основні наукові положення, які виносяться на захист,
із зазначенням відмінності одержаних результатів від відомих раніше.

% За наявності у вступі можуть також вказуватися
\paragraph{Практичне значення отриманих результатів}

Надаються відомості про використання результатів досліджень
або рекомендації щодо їх практичного використання.

\paragraph{Особистий внесок здобувача}

Якщо у дисертації використано ідеї або розробки,
що належать співавторам, разом з якими здобувачем опубліковано наукові праці,
обов'язково зазначається
конкретний особистий внесок здобувача в такі праці або розробки;
здобувач має також додати посилання на дисертації співавторів,
у яких було використано результати спільних робіт.

% Апробація матеріалів дисертації
\begin{approval}
У вступі подається апробація матеріалів дисертації
(зазначаються назви конференції, конгресу, симпозіуму, семінару, школи,
місце та дата проведення).

Основні результати дослідження доповідалися на наукових
конференціях різного рівня та наукових семінарах. Це такі
конференції:
\begin{itemize}
\item Український математичний конгрес, Київ, 21--23~серпня
2001~р. \participation{секційна доповідь};
% Команда \participation працює так,
% що форму участі у вступі не буде показувати,
% але якщо цей текст перенести в додаток, то буде.

\item Звітна конференція викладачів, аспірантів та докторантів університету,
  Київ, 1--2~лютого 2002~р. \participation{пленарна доповідь};

\item Конференція молодих вчених <<Сучасна алгебра і топологія>>,
  Одеса, 15--20~серпня 2003~р. \participation{стендова доповідь};

\item \ldots
\end{itemize}
Це такі семінари:
\begin{itemize}
\item семінар відділу теорії функцій Інституту математики НАН
України (керівник: чл.-кор. НАН України О.",І.",Степанець);
% Говорити про форму участі в семінарі немає сенсу, на мій погляд.
% (Якщо розуміти семінар як такий науковий захід,
% де тільки один доповідач (або небагато), а всі інші "--- слухачі.
% Тоді не може бути ніякої пленарної, секційної чи стендової доповіді,
% як на конференції, де може бути кілька паралельних потоків.
% Можливо, в інших науках семінаром називають щось інше.)
% Але за потреби команду \participation тут теж можна використовувати.

\item \ldots
\end{itemize}
\end{approval}

\paragraph{Структура та обсяг дисертації}

Анонсується структура дисертації, зазначається її загальний обсяг.
% Вступ
%%
%% This is file `xampl-ch1.tex',
%% generated with the docstrip utility.
%%
%% The original source files were:
%%
%% vakthesis.dtx  (with options: `xampl-ch1')
%% 
%% IMPORTANT NOTICE:
%% 
%% For the copyright see the source file.
%% 
%% Any modified versions of this file must be renamed
%% with new filenames distinct from xampl-ch1.tex.
%% 
%% For distribution of the original source see the terms
%% for copying and modification in the file vakthesis.dtx.
%% 
%% This generated file may be distributed as long as the
%% original source files, as listed above, are part of the
%% same distribution. (The sources need not necessarily be
%% in the same archive or directory.)
%% xampl-ch1.tex  Приклад розділу дисертації
% Приклад назви розділу і мітки, на яку можна посилатися в тексті
\chapter{Подання дійсних чисел рядами~Остроградського $1$-го виду}
\label{ch:o1series}

Це не є справжній розділ дисертації. Це лише приклад, який повинен
допомогти користувачу підготувати свій файл. Але я зробив його з
розділу~1 своєї дисертації.

У цьому розділі вивчається розвинення дійсного числа у
знакозмінний ряд спеціального вигляду, який називається рядом
Остроградського $1$-го виду.

Досліджуються тополого"=метричні та фрактальні властивості множини
неповних сум заданого ряду Остроградського $1$-го виду, а також
властивості розподілів ймовірностей на множині неповних сум.


% Приклад назви підрозділу
\section{Означення ряду Остроградського $1$-го виду}

% Приклад означення
\begin{definition}
% Приклад виноски (\footnote)
\emph{Рядом Остроградського $1$-го виду}\footnote{Далі часто
будемо називати просто \emph{рядом Остроградського}, оскільки ми
не досліджуємо ряди Остроградського $2$-го виду.} називається
скінченний або нескінченний вираз вигляду
% Приклад формули з номером
\begin{equation}\label{eq:o1series}
\frac1{q_1}-\frac1{q_1q_2}+\dots +\frac{(-1)^{n-1}}{q_1q_2\dots
q_n}+\dotsb,
\end{equation}
де $q_n$ "--- натуральні числа і $q_{n + 1}>q_n$ для будь-якого
$n\in\N$. Числа $q_n$ називаються \emph{елементами ряду
Остроградського $1$-го виду}.
\end{definition}

Число елементів може бути як скінченним, так і нескінченним.
У~першому випадку будемо записувати ряд Остроградського у вигляді
% Приклад формули без номера
\[
\frac1{q_1}-\frac1{q_1q_2}+\dots +\frac{(-1)^{n-1}}{q_1q_2\dots
q_n}
\]
або скорочено
\begin{equation*}
\Osign1(q_1,q_2,\dots,q_n)
\end{equation*}
і називати скінченним рядом Остроградського або
$n$\nobreakdash-\hspace{0pt}елементним рядом Остроградського; а в
другому випадку будемо записувати ряд Остроградського у вигляді
% Приклад посилання на формулу
\eqref{eq:o1series} або скорочено
\begin{equation*}
\Osign1(q_1,q_2,\dots,q_n,\dots)
\end{equation*}
і називати нескінченним рядом Остроградського.


\section{Означення та властивості підхідних чисел}

\begin{definition}\label{def:convergent}
\emph{Підхідним числом порядку $k$} ряду Остроградського $1$-го
виду називається раціональне число
\[
\frac{A_k}{B_k} = \frac{1}{q_1}-\frac{1}{q_1q_2}+\dots
+\frac{(-1)^{k-1}}{q_1q_2\dots q_k} = \Osign1(q_1,q_2,\dots,q_k).
\]
\end{definition}

Зрозуміло, що $n$-елементний ряд Остроградського має $n$ підхідних
чисел, причому підхідне число $n$-го порядку $\frac{A_n}{B_n}$
збігається зі значенням цього ряду Остроградського.

% Приклад теореми
\begin{theorem}\label{th:convergents}
Для будь-якого натурального $k$ правильні формули
\begin{equation}\label{eq:convergents}
\left\{
\begin{aligned}
&A_k=A_{k-1}q_k+(-1)^{k-1},\\
&B_k=B_{k-1}q_k=q_1q_2\dots q_k
\end{aligned}
\right.
\end{equation}
\textup(якщо покласти, що $A_0=0$, $B_0=1$\textup).
\end{theorem}

% Приклад доведення
\begin{proof}
Проведемо доведення методом математичної індукції по~$k$. Для
$k=1$ формули правильні. Справді,
\[
\frac{A_1}{B_1}=\frac{1}{q_1}=\frac{A_0q_1+(-1)^0}{B_0q_1}.
\]

Припустимо, що формули \eqref{eq:convergents} правильні для
деякого $k=m$, тобто
\begin{align*}
\left\{
\begin{aligned}
&A_m=A_{m-1}q_m+(-1)^{m-1},\\
&B_m=B_{m-1}q_m=q_1q_2\ldots q_m,
\end{aligned}
\right.
\end{align*}
і доведемо ці формули для $k=m+1$. Маємо
% Приклад формули, що займає більше одного рядка.
% Оточення align, рядки вирівняні по знаку =
\begin{align*}
\frac{A_{m+1}}{B_{m+1}}&=\frac{1}{q_1}-\frac{1}{q_1 q_2}+\dots
+\frac{(-1)^{m-1}}{q_1q_2\dots q_m}+\frac{(-1)^m}{q_1q_2\dots
q_mq_{m + 1}}=\\ &=\frac{A_m}{B_m}+\frac{(-1)^m}{B_mq_{m + 1}} =
\frac{A_mq_{m+1}+(-1)^m}{B_mq_{m + 1}}.
\end{align*}

Отже, за принципом математичної індукції формули
\eqref{eq:convergents} правильні для будь"=якого натурального $k$.
\end{proof}

% Приклад леми
\begin{lemma}
Для будь-якого натурального $k$ правильна рівність
\begin{equation}\label{eq:convergents1}
\frac{A_{k-1}}{B_{k-1}}-\frac{A_k}{B_k}=\frac{(-1)^k}{B_k}.
\end{equation}
\end{lemma}

\begin{lemma}
Для будь-якого натурального $k\geq2$ правильна рівність
\begin{equation}\label{eq:convergents2}
\frac{A_{k-2}}{B_{k-2}}-\frac{A_k}{B_k}=\frac{(-1)^{k-1}(q_k-1)}{B_k}.
\end{equation}
\end{lemma}


\section{Розклад числа у знакозмінний ряд за $1$-м алгоритмом Остроградського}

Почнемо з геометричної ілюстрації алгоритму. Нехай маємо відрізки
$A$ та $B$, $A<B$. Щоб застосувати $1$-й алгоритм Остроградського
до числа $\frac{A}{B}$, будемо відкладати відрізок $A$ на відрізку
$B$, поки не отримаємо залишок $A_1<A$ (див.
% Приклад посилання на малюнок
рис.~\ref{fig:o1alg}). Нехай відрізок $A$ вміщується $q_1$ разів у
відрізку $B$, тоді
\[
B=q_1A+A_1.
\]
Далі відкладемо відрізок $A_1$ не на меншому відрізку $A$ (як у
алгоритмі Евкліда), а на тому ж відрізку $B$ до отримання залишку
$A_2<A_1$. Нехай відрізок $A_1$ вміщується $q_2$ разів у відрізку
$B$, тоді
\[
B=q_2A_1+A_2.
\]
Відкладаючи відрізок $A_2$ знову на відрізку $B$ і~т.~д. до
нескінченності або до отримання нульового залишку, будемо мати
\begin{align*}
&B=q_3A_2+A_3,\\
&B=q_4A_3+A_4
\end{align*}
і~т.~д. З отриманих рівностей випливає, що має місце розклад
\[
\frac AB = \frac{1}{q_1} - \frac{1}{q_1q_2} + \frac{1}{q_1q_2q_3}
- \frac{1}{q_1q_2q_3q_4} + \dotsb,
\]
і тут, як легко бачити,
\[
q_1<q_2<q_3<q_4<\dotsb.
\]

% Приклад малюнка
\begin{figure}[htbp]
\setlength{\unitlength}{1mm}
\begin{center}
\begin{picture}(105,35)
\put(6,28){$A$}\put(51,28){$A_1$} \put(0,25){\line(1,0){58}}
\multiput(0,24)(16,0){4}{\line(0,1){2}}\put(58,24){\line(0,1){2}}
\put(28,19){$B$} \put(75,25){$B=3A+A_1$}
\put(3,9){$A_1$}\put(52,9){$A_2$} \put(0,6){\line(1,0){58}}
\multiput(0,5)(10,0){6}{\line(0,1){2}}\put(58,5){\line(0,1){2}}
\put(28,0){$B$} \put(75,6){$B=5A_1+A_2$}
\end{picture}
\caption{Геометрична ілюстрація $1$-го алгоритму Остроградського:
тут відрізок $A$ вміщується 3~рази у відрізку $B$, відрізок $A_1$
вміщується 5~разів у відрізку $B$ і~т.~д.}
\label{fig:o1alg}
\end{center}
\end{figure}

Таким чином, \emph{$1$-й алгоритм Остроградського} розкладу
дійсного числа $x\in(0,1)$ у знакозмінний ряд полягає в
наступному.

\begin{description}
\item[Крок~1.] Покласти $\alpha_0=x$, $i=1$.

\item[Крок~2.] Знайти такі числа $q_i$ та $\alpha_i$, що
\[
1=q_i\alpha_{i-1}+\alpha_i \quad \text{і} \quad
0\leq\alpha_i<\alpha_{i-1}.
\]

\item[Крок~3.] Якщо $\alpha_i=0$, то припинити обчислення. Інакше
"--- збільшити $i$ на $1$ та перейти до кроку~2.
\end{description}

\begin{theorem}\label{thm:ostrogradsky}
Кожне дійсне число $x\in(0,1)$ можна подати у вигляді ряду
Остроградського $1$-го виду~\eqref{eq:o1series}. Причому, якщо
число $x$ ірраціональне, то це можна зробити єдиним чином і
вираз~\eqref{eq:o1series} має при цьому нескінченне число
доданків; якщо ж число $x$ раціональне, то його можна подати у
вигляді~\eqref{eq:o1series} зі скінченним числом доданків двома
різними способами:
\[
x=\Osign1(q_1,q_2,\dots,q_{n-1},q_n)=\Osign1(q_1,q_2,\dots,q_{n-1},q_n-1,q_n).
\]
\end{theorem}

% Приклад посилання на таблицю
У табл.~\ref{tab:ellipse.hyperbola.parabola} наведені деякі
формули для еліпса, гіперболи і параболи.

% Приклад таблиці
\begin{table}[htbp]
\caption{Еліпс, гіпербола і парабола. Деякі формули}
\label{tab:ellipse.hyperbola.parabola}
\begin{tabularx}{\textwidth}{|X|c|c|c|}
\hline
                   & Еліпс                                    & Гіпербола                                & Парабола          \\
\hline
Канонічне рівняння & $\frac{x^2}{a^2}+\frac{y^2}{b^2}=1$      & $\frac{x^2}{a^2}-\frac{y^2}{b^2}=1$      & $y^2=2px$         \\
Ексцентриситет     & $\varepsilon=\sqrt{1-\frac{b^2}{a^2}}<1$ & $\varepsilon=\sqrt{1+\frac{b^2}{a^2}}>1$ & $\varepsilon=1$   \\
Фокуси             & $(a\varepsilon,0)$, $(-a\varepsilon,0)$  & $(a\varepsilon,0)$, $(-a\varepsilon,0)$  & $(\frac{p}{2},0)$ \\
\hline
\multicolumn{4}{|l|}{Корн~Г., Корн~Т. Справочник по математике. М., 1974. С.~72.} \\
\hline
\end{tabularx}
\end{table}


\section{Множина неповних сум ряду Остроградського та розподіли ймовірностей на ній}

Візьмемо довільну \emph{фіксовану} послідовність $\{q_k\}$
натуральних чисел з умовою $q_{k+1}>q_k$ для всіх $k\in\N$ і
розглянемо їй відповідний ряд Остроградського $1$-го
виду~\eqref{eq:o1series} з сумою $r$. Число $r$ можна записати у
вигляді
\begin{equation}
r=d-b, \quad \text{де} \quad d=\sum_{i=1}^\infty
\frac1{q_1q_2\dots q_{2i-1}}, \quad b=\sum_{i=1}^\infty
\frac1{q_1q_2\dots q_{2i}}.
\end{equation}

% Приклад назви пункту
\subsection{Тополого-метричні та фрактальні властивості множини
неповних сум ряду Остроградського}

\emph{Циліндром} рангу $m$ з основою $c_1c_2\dots c_m$ називається
множина $\Delta'_{c_1c_2\dots c_m}$ всіх неповних сум, які мають
зображення $\Delta_{c_1c_2\dots c_ma_{m+1}\dots a_{m+k}\dots}$, де
$a_{m+j}\in\set{0,1}$ для будь-якого $j\in\N$. Очевидно, що
\[
\Delta'_{c_1c_2\dots c_ma}\subset\Delta'_{c_1c_2\dots c_m}, \quad
a\in\set{0,1}.
\]

% Приклад теоремоподібної структури з додатковою інформацією в заголовку
\begin{definition}[{\cite[с.~59]{Pra98}}]
\emph{Фракталом} називається кожна континуальна обмежена множина
простору $\R^1$, яка має тривіальну (рівну $0$ або $\infty$)
$H_\alpha$-міру Хаусдорфа, порядок $\alpha$ якої дорівнює
топологічній розмірності.
\end{definition}

Ті нуль-множини Лебега простору $\R^1$, розмірність
Хаусдорфа\nobreakdash--\hspace{0pt}Безиковича яких дорівнює $1$,
називаються \emph{суперфракталами}, а континуальні множини, що
мають нульову розмірність Хаусдорфа--Безиковича, називаються
\emph{аномально фрактальними}.


% Приклад висновків до розділу
\section*{Висновки до розділу~\ref{ch:o1series}}

У розділі~\ref{ch:o1series} введене поняття ряду Остроградського
$1$-го виду та його підхідних чисел, запропоновані деякі
властивості підхідних чисел. Доведено, що кожне дійсне число
$x\in(0,1)$ можна подати у вигляді ряду Остроградського $1$-го
виду: ірраціональне "--- єдиним чином у вигляді нескінченного ряду
Остроградського, раціональне "--- двома різними способами у
вигляді скінченного ряду Остроградського. Ці результати не є
новими, їх можна знайти, наприклад, у
роботах~\cite{Rem51,Sie11STNW,Pie29,VaZ75,Sha86} та~ін. Вони
наведені тут для повноти викладу.

Новими в цьому розділі є результати, що стосуються неповних сум
ряду Остроградського. Описані тополого-метричні та фрактальні
властивості множини неповних сум ряду Остроградського. Описано
множини чисел, ряди Остроградського яких є простими і густими
відповідно. Доведено, що випадкова неповна сума ряду
Остроградського має або дискретний розподіл або сингулярний
розподіл канторівського типу. Досліджено поведінку на
нескінченності модуля характеристичної функції випадкової неповної
суми ряду Остроградського.
%        Розділ 1
%\include{xampl-ch2}%       Розділ 2
%\include{xampl-ch3}%       Розділ 3
%\include{xampl-ch4}%       Розділ 4 і т. д. ще скільки потрібно розділів
%%
%% This is file `xampl-concl.tex',
%% generated with the docstrip utility.
%%
%% The original source files were:
%%
%% vakthesis.dtx  (with options: `xampl-concl')
%% 
%% IMPORTANT NOTICE:
%% 
%% For the copyright see the source file.
%% 
%% Any modified versions of this file must be renamed
%% with new filenames distinct from xampl-concl.tex.
%% 
%% For distribution of the original source see the terms
%% for copying and modification in the file vakthesis.dtx.
%% 
%% This generated file may be distributed as long as the
%% original source files, as listed above, are part of the
%% same distribution. (The sources need not necessarily be
%% in the same archive or directory.)
%% xampl-concl.tex  Приклад висновків до дисертації
\chapter*{Висновки}

Це не є справжні висновки до дисертації. Це лише приклад, який
повинен допомогти користувачу підготувати свій файл. Але я зробив
його з висновків до своєї дисертації.

Ряди Остроградського $1$-го виду дозволяють розширити можливості
формального задання і аналітичного дослідження фрактальних множин,
сингулярних мір, недиференційовних функцій та інших об'єктів зі
складною локальною будовою.

В дисертаційній роботі отримано такі результати.
\begin{itemize}
\item Розроблено основи метричної теорії чисел, представлених
рядами Остроградського $1$-го виду. Зокрема, досліджено геометрію
розвинень чисел в ряди Остроградського $1$-го виду, отримано
основне метричне відношення та його оцінки, які допомагають у
розв'язанні задач про міру Лебега множин чисел з умовами на
елементи зображення.

\item Знайдено умови нуль-мірності (додатності міри) певних класів
замкнених ніде не щільних множин чисел, заданих умовами на
елементи їх розвинення в ряд Остроградського $1$-го виду.

\item Вивчено тополого-метричні та фрактальні властивості множини
неповних сум заданого ряду Остроградського $1$-го виду та
розподілів ймовірностей на ній.

\item Досліджено структуру та властивості випадкової величини з
незалежними різницями послідовних елементів її представлення рядом
Остроградського $1$-го виду.

\item Вивчено диференціальні та фрактальні властивості однієї
функції, заданої перетворювачем елементів ряду Остроградського
$1$-го виду її аргумента в двійкові цифри значення функції.
\end{itemize}

Як виявилося, існують принципові відмінності метричної теорії
рядів Остроградського та метричної теорії ланцюгових дробів.
Зокрема, існує клас замкнених ніде не щільних множин додатної міри
Лебега, описаних в термінах елементів ряду Остроградського. В той
же час, аналогічні множини, задані у термінах елементів
ланцюгового дробу, мають нульову міру Лебега.

Проведені дослідження лежать в руслі сучасних математичних
досліджень об'єктів зі складною локальною поведінкою (будовою),
пов'язаних з ланцюговими дробами, рядами Люрота,
$\beta$-розкладами тощо, інтерес до яких у світі достатньо
високий. Отримані результати та запропоновані методи можуть бути
корисними при розв'язанні задач метричної теорії чисел,
представлених рядами Остроградського $2$-го виду або іншими
зображеннями з нескінченним алфавітом.
%      Висновки
%% mon2017dev-bib.tex  Приклад списку використаних джерел (для mon2017dev.tex)

% TODO: Це потрібно для статті Sie11STNW з файла xampl-thesis.bib.
% Ці команди жорстко прописані у полі note.
% Треба зробити зразок без цього.
\def\selectlanguageifdefined#1{
\expandafter\ifx\csname date#1\endcsname\relax
\else\selectlanguage{#1}\fi}
\providecommand*{\BibEmph}[1]{#1}

% Рекомендований перелік стилів оформлення списку наукових публікацій
% Додаток 3 до Вимог до оформлення дисертації (пункт 11 розділу ІІІ)
% https://zakon.rada.gov.ua/go/z0155-17

% ------------------------------------------------------------------------------
% Стиль оформлення		BibTeX-стиль		Джерело
% списку наукових публікацій
% ------------------------------------------------------------------------------
% ДСТУ 8302:2015		dstu2015*		https://codeberg.org/mdmisch/dstu
% 				gost2008*		https://ctan.org/pkg/gost
% MLA (Modern Language		mla			https://ctan.org/pkg/mla
%   Association)
% APA (American Psychological	apalike			https://ctan.org/pkg/bibtex
%   Association)[1][2]		apacite			https://ctan.org/pkg/apacite
% Chicago/Turabian[1]		achicago		https://ctan.org/pkg/achicago-bst
% Harvard[1]			[3]			https://ctan.org/pkg/harvard
% ACS (American			achemso/biochem		https://ctan.org/pkg/achemso
%   Chemical Society)
% AIP (American			aipauth4-2/aipnum4-2	https://ctan.org/pkg/revtex
%   Institute of Physics)
% IEEE (Institute of Electrical	ieeetr			https://ctan.org/pkg/bibtex
%   and Electronics Engineers)	IEEEtran*		https://ctan.org/pkg/ieeetran
% Vancouver[1]			[3]			https://ctan.org/pkg/vancouver
% OSCOLA			???
% APS (American			apsrev4-2/apsrmp4-2	https://ctan.org/pkg/revtex
%   Physical Society)[1]
% Springer MathPhys[1]		spmpsci			https://resource-cms.springernature.com/springer-cms/rest/v1/content/25980412/data/v2
% ------------------------------------------------------------------------------

% [1] Оригінальне гіперпосилання в додатку 3 вже недоступне.
% Інструкції від Springer щодо підготовки рукописів
% (зокрема щодо оформлення списку посилань) тепер на цій сторінці:
% https://www.springer.com/gp/authors-editors/book-authors-editors/your-publication-journey/manuscript-preparation
% [2] Оригінальне гіперпосилання в додатку 3 вже переадресовує на іншу сторінку.
% Інструкції від Elsevier щодо оформлення списку посилань
% у журналі «Learning and Instruction» тепер доступні тут:
% https://www.sciencedirect.com/journal/learning-and-instruction/publish/guide-for-authors#68000
% [3] Harvard і Vancouver — це радше системи оформлення посилань
% (у текстовій і числовій формі відповідно),
% ніж конкретні стилі оформлення списку наукових публікацій.
% Але на CTAN є пакунки з такими іменами, які містять кілька BibTeX-стилів.
% Можливо, вони будуть корисні у зв'язку з рекомендаціями додатку 3.

% Корисний огляд різноманітних BibTeX-стилів від Reed College:
% https://www.reed.edu/it/help/LaTeX/bibtexstyles.html

\begin{bibset}{Список використаних джерел}
  \bibliographystyle{ieeetr}% або інший BibTeX-стиль, наприклад, gost2008s
  %
  % Якщо не треба нумерація з крапкою, можна закоментувати наступні три рядки.
  \makeatletter
  \renewcommand\@biblabel[1]{#1.}
  \makeatother
  \bibliography{xampl-thesis,xampl-mybib}
\end{bibset}
%   Список використаних джерел (який включає список публікацій здобувача)
\appendix
%% mon2017dev-app1.tex  Приклад додатку зі списком публікацій та відомостями про апробацію (для mon2017dev.tex)

\chapter{Список публікацій здобувача за темою дисертації
  та~відомості~про апробацію результатів дисертації}

Обов'язковим додатком до дисертації є
список публікацій здобувача за темою дисертації
та відомості про апробацію результатів дисертації
(зазначаються назви конференції, конгресу, симпозіуму, семінару, школи,
місце та дата проведення, форма участі).

\repeatauthorpublications

% TODO: Перевірити, якщо головна мова документа -- english.
\repeatapproval
%  Додаток 1 (список публікацій та відомості про апробацію)
%\include{xampl-app2}%      Додаток 2 і т. д. ще скільки потрібно додатків

\end{document}
