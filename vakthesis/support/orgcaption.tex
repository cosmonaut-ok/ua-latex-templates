%%% orgcaption.tex --- Відновлення оригінального заголовка таблиць

% Класи vakthesis/vakaref жорстко прописують оформлення заголовка
% таблиці в командах \@maketablecaption, \LT@makecaption тощо.
% Причому, деякі зміни зроблено за допомогою \AtBeginDocument, тобто
% зміни, які користувач міг зробити за допомогою сторонніх пакунків,
% будуть проігноровані.  Через це, зокрема, неможливо використовувати
% пакунок caption з цими класами.  Можливо, найпростіший спосіб
% уникнути цього конфлікту — відкинути зміни, які роблять класи
% vakthesis/vakaref в оформленні таблиць, і використовувати для цього
% можливості caption.

% vlamor запитував про це 2015/01/27.
% (http://linux.org.ua/cgi-bin/yabb/YaBB.pl?num=1229461517/224#224).

% Потрібний код користувач має взяти з цього файла і помістити у
% преамбулу свого документа або у свій .sty-файл.  Цей файл є лише
% зразком і не може використовуватися безпосередньо з класами
% vakthesis/vakaref.

% Перевірено з vakthesis/vakaref v0.08 від 2009/04/01.

\documentclass{vakthesis}
% \documentclass{vakaref}

\usepackage[T2A]{fontenc}
\usepackage[cp1251]{inputenc}
\usepackage[ukrainian]{babel}

\usepackage{tabularx}
\usepackage{longtable}
\usepackage[labelsep=endash]{caption}

\makeatletter
% Зберігаємо ті означення команд, які були на час виклику caption
\let\ORG@makecaption\@makecaption
\let\ORGlongtable\longtable
\let\ORGLT@makecaption\LT@makecaption
% «На початку документа», після змін, які зробили і клас, і пакунки,
% відновлюємо означення команд, які були на час виклику caption
\AtBeginDocument{%
  \let\@maketablecaption\ORG@makecaption
  \let\longtable\ORGlongtable
  \let\LT@makecaption\ORGLT@makecaption
}
\makeatother

\begin{document}

% Зразок рухомої таблиці
% Змінений зразок з документації vakthesis
\begin{table}[htbp]
\caption{Назва таблиці}
\begin{tabularx}{\textwidth}{|X|c|c|c|}
\hline
                  &Еліпс&Гіпербола&Парабола\\
\hline
Канонічне рівняння&1    &2        &3       \\
Ексцентриситет    &4    &5        &6       \\
Фокуси            &7    &8        &9       \\
\hline
\end{tabularx}
\end{table}

% Зразок довгої таблиці (може простягатися більш ніж на одну сторінку)
% Змінений зразок з документації longtable
\begin{longtable}{@{*}r||p{1in}@{*}}
\caption[An optional table caption ...]{A long table\label{long}}\\
\hline\hline
\multicolumn{2}{@{*}c@{*}}%
{This part appears at the top of the table}\\
\textsc{First}&\textsc{Second}\\
\hline\hline
\endfirsthead
\caption[]{(continued)}\\
\hline\hline
\multicolumn{2}{@{*}c@{*}}%
{This part appears at the top of every other page}\\
\textbf{First}&\textbf{Second}\\
\hline\hline
\endhead
\hline
This goes at the&bottom.\\
\hline
\endfoot
\hline
These lines will&appear\\
in place of the & usual foot\\
at the end& of the table\\
\hline
\endlastfoot
longtable columns are specified& in the \\
same way as in the tabular& environment.\\
\hline
\multicolumn{2}{||c||}{This is a ...}\\
\hline
Some lines may take...&
\raggedleft This last column is a ``p'' column...
\tabularnewline
\hline
Lots of lines& like this.\\
\hline
Lots\footnote{...} of lines& like this.\\
Lots of lines& like this\footnote{...}\\
\hline
Lots of lines& like this.\\
\hline
\end{longtable}

\end{document}
