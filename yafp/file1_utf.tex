\documentclass{report}

\usepackage[T2A]{fontenc}
\usepackage[cp1251]{inputenc}
\usepackage[ukrainian]{babel}

\usepackage{amsmath}
\usepackage{amsthm}
%\usepackage{hyperref}

\usepackage{yafp}

\theoremstyle{plain}
\newtheorem{theorem}{Theorem}[chapter]
\theoremstyle{remark}
\newtheorem{lemma}[theorem]{Lemma}%[chapter]

%\def\eqref#1{(\ref{#1})}

\begin{document}

\chapter*{Intro}

\begin{equation}\label{eq:intro1}
f(x)=a_0+a_1x
\end{equation}

\begin{theorem}\label{thm:intro1}
Start start start start start.
\end{theorem}

EXCERPT BEGINS HERE.\hrulefill

%%%%%%%%%%%%%%%%%%%%%%%%%%%%%%%%%%%%%%%%%%%%%%%%%%%%%%%%%%%%%%%%%%%%%%%%
\begin{yafea}

\begin{equation}\label{eq:equation}
x^2 \neq x^3
\end{equation}

\begin{lemma}\label{lem:bbb}
Text text text text, text text text text text text. Text text text text
text text text text text.
\end{lemma}

\begin{theorem}\label{thm:two}
Bla-bla-bla...
\begin{equation}\label{eq:eqinthm}
\sum_{k=1}^\infty a_k = \frac\pi2,
\end{equation}
Bla-bla-bla...
\end{theorem}

\begin{theorem}\label{thm:five}
This is another great theorem.
\end{theorem}

From Theorem~\ref{thm:two} it follows...

\begin{theorem}[Hardy]\label{thm:Hardy}
If $x=y$, then~\eqref{eq:eqinthm} holds.
\end{theorem}

Оточення з пакета \texttt{amsmath}:
gather
\begin{gather}
\label{eq:gather1}
a_x=b_{11}+c_{11}\\
\label{eq:gather2}
a_y=b_{21}+c_{21}+d_{21}\\
\label{eq:gather3}
a_z=b_{31}+c_{31}
\end{gather}
align
\begin{align}
\label{eq:align1}
a_{11}&=b_{11}, & a_{12}&=b_{12}\\
\label{eq:align2}
a_{21}&=b_{21}, & a_{22}&=b_{22}+c^2
\end{align}
flalign
\begin{flalign}
\label{eq:flalign1}
a_{11}&=b_{11}, & a_{12}&=b_{12}\\
\label{eq:flalign2}
a_{21}&=b_{21}, & a_{22}&=b_{22}+c^2
\end{flalign}

Плаваючі об'єкти десь тут починаються...

\begin{figure}[htbp]
\emph{Тут буде другий малюнок! Дуже гарний! Кольоровий!}
\caption{Назва малюнка 2}\label{fig:figure2}
\end{figure}

\begin{table}[htbp]
\caption{Назва таблиці 1}\label{tab:table1}
\emph{Тіло першої таблиці}
\end{table}

\end{yafea}
%%%%%%%%%%%%%%%%%%%%%%%%%%%%%%%%%%%%%%%%%%%%%%%%%%%%%%%%%%%%%%%%%%%%%%%%

EXCERPT ENDS HERE.\hrulefill

\begin{equation}\label{eq:intro2}
f(x)=a_0+a_1x+a_2x^2
\end{equation}

\begin{theorem}\label{thm:intro2}
Finish finish finish finish finish.
\end{theorem}


\chapter{Формули стандартного \LaTeX{} і \texttt{amsmath}}

Перша формула.
\begin{equation}\label{eq:ch1first}
x_1
\end{equation}

Формула тут
\begin{equation}\label{eq:equation}
x^2 \neq x^3
\end{equation}

Послідовність формул:
\begin{eqnarray}
\label{eq:eqnarray1}
\sin\alpha^2+\cos\alpha^2=1\\
\label{eq:eqnarray2}
\sin\alpha^2+\cos\alpha^2\neq2
\end{eqnarray}
З формул~\eqref{eq:eqnarray1} та~\eqref{eq:eqnarray2}...

\begin{equation}\label{eq:array}
\begin{array}{lllll}
x & 2 & 3 & 4 & 5 \\
y & 7 & 8 & 9 & 10 \\
\end{array}
\end{equation}

Далі \texttt{amsmath}, align.
\begin{align}
\label{eq:align1}
a_{11}&=b_{11}, & a_{12}&=b_{12}\\
\label{eq:align2}
a_{21}&=b_{21}, & a_{22}&=b_{22}+c^2
\end{align}
text
\begin{align*}
a_{11}&=b_{11}, & a_{12}&=b_{12}\\
a_{21}&=b_{21}, & a_{22}&=b_{22}+c^2
\end{align*}
gather
\begin{gather}
\label{eq:gather1}
a_x=b_{11}+c_{11}\\
\label{eq:gather2}
a_y=b_{21}+c_{21}+d_{21}\\
\label{eq:gather3}
a_z=b_{31}+c_{31}
\end{gather}
flalign
\begin{flalign}
\label{eq:flalign1}
a_{11}&=b_{11}, & a_{12}&=b_{12}\\
\label{eq:flalign2}
a_{21}&=b_{21}, & a_{22}&=b_{22}+c^2
\end{flalign}
multline
\begin{multline}\label{eq:multline}
x=\frac1{q_1}-\frac1{q_1q_2}+\frac1{q_1q_2q_3}-\frac1{q_1q_2q_3q_4}+\dots
  +\frac{(-1)^{k-1}}{q_1q_2q_3q_4\dots q_k}+{}\\
 =\Delta_{q_1,q_2,q_3,q_4,\dots,q_k,\dots}
\end{multline}
alignat
\begin{alignat}{2}
\label{eq:alignat1}
a_{11}&=b_{11}, &\quad& a_{12}=b_{12}\\
\label{eq:alignat2}
a_{21}&=b_{21}, && a_{22}=b_{22}+c^2
\end{alignat}
split
\begin{equation}\label{eq:split}
\begin{split}
x^2 &= x^3-x^4\\
y^2 &= y^3-y^4+z^5
\end{split}
\end{equation}
aligned
\begin{equation}\label{eq:aligned}
\left.
\begin{aligned}
x^2 &= x^3-x^4\\
y^2 &= y^3-y^4+z^5
\end{aligned}
\right\}
\qquad \text{Maxwell's equations}
\end{equation}
cases
\begin{equation}\label{eq:cases}
|x|=
\begin{cases}
x, & \text{якщо $x\geq0$},\\
-x, & \text{якщо $x<0$}.
\end{cases}
\end{equation}

Остання формула.
\begin{equation}\label{eq:ch1last}
x_n
\end{equation}


\chapter{Теореми}

\begin{theorem}\label{thm:one}
This is a great theorem.
\end{theorem}

\begin{theorem}\label{thm:two}
Bla-bla-bla...
\begin{equation}\label{eq:eqinthm}
\sum_{k=1}^\infty a_k = \frac\pi2,
\end{equation}
Bla-bla-bla...
\end{theorem}

From Theorem~\ref{thm:two} it follows...

\begin{theorem}[Hardy]\label{thm:Hardy}
If $x=y$, then~\eqref{eq:eqinthm} holds.
\end{theorem}

\begin{lemma}\label{lem:aaa}
Bla-bla-bla...
\end{lemma}

\begin{theorem}\label{thm:five}
This is another great theorem.
\end{theorem}

\begin{lemma}\label{lem:bbb}
Text text text text, text text text text text text. Text text text text
text text text text text.
\end{lemma}


\chapter{Плаваючі об'єкти}

У цьому розділі "--- приклади плаваючих об'єктів.

\begin{figure}[htbp]
\emph{Тут буде перший малюнок!}
\caption{Назва малюнка 1}\label{fig:figure1}
\end{figure}

Текст текст текст, текст текст текст текст текст, текст текст текст.

\begin{table}[htbp]
\caption{Назва таблиці 1}\label{tab:table1}
\emph{Тіло першої таблиці}
\end{table}

Текст текст текст текст текст текст, текст текст текст.

\begin{table}[htbp]
\caption{Назва таблиці 2}\label{tab:table2}
\emph{Тіло другої таблиці}
\end{table}

Текст текст текст, текст текст, а текст текст текст текст.

\begin{figure}[htbp]
\emph{Тут буде другий малюнок! Дуже гарний! Кольоровий!}
\caption{Назва малюнка 2}\label{fig:figure2}
\end{figure}

І на завершення ще трохи тексту.

\end{document}
