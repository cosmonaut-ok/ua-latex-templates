\documentclass{report}

\usepackage[T2A]{fontenc}
\usepackage[cp1251]{inputenc}
\usepackage[ukrainian]{babel}

\usepackage{amsmath}
\usepackage{amsthm}
%\usepackage{hyperref}

\usepackage{yafp}
\usepackage{xr}
\externaldocument{file1}

\swapnumbers
%\theoremstyle{plain}
\newtheorem{theorem}{Theorem}[chapter]
%\theoremstyle{remark}
\newtheorem{lemma}[theorem]{Lemma}%[chapter]

%\def\eqref#1{(\ref{#1})}

\begin{document}

З формул~\eqref{eq:eqnarray1} та~\eqref{eq:eqnarray2} випливає
теорема~\ref{thm:Hardy}.

%\chapter*{Intro}

\begin{equation}\label{eq:aref.intro1}
f(x)=a_0+a_1x
\end{equation}

\begin{theorem}\label{thm:aref.intro1}
Start start start start start.
\end{theorem}

EXCERPT BEGINS HERE.\hrulefill

%%%%%%%%%%%%%%%%%%%%%%%%%%%%%%%%%%%%%%%%%%%%%%%%%%%%%%%%%%%%%%%%%%%%%%%%
\begin{yafea}

\begin{equation}\label{eq:equation}
x^2 \neq x^3
\end{equation}

\begin{lemma}\label{lem:bbb}
Text text text text, text text text text text text. Text text text text
text text text text text.
\end{lemma}

\begin{theorem}\label{thm:two}
Bla-bla-bla...
\begin{equation}\label{eq:eqinthm}
\sum_{k=1}^\infty a_k = \frac\pi2,
\end{equation}
Bla-bla-bla...
\end{theorem}

\begin{theorem}\label{thm:five}
This is another great theorem.
\end{theorem}

From Theorem~\ref{thm:two} it follows...

\begin{theorem}[Hardy]\label{thm:Hardy}
If $x=y$, then~\eqref{eq:eqinthm} holds.
\end{theorem}

Оточення з пакета \texttt{amsmath}:
gather
\begin{gather}
\label{eq:gather1}
a_x=b_{11}+c_{11}\\
\label{eq:gather2}
a_y=b_{21}+c_{21}+d_{21}\\
\label{eq:gather3}
a_z=b_{31}+c_{31}
\end{gather}
align
\begin{align}
\label{eq:align1}
a_{11}&=b_{11}, & a_{12}&=b_{12}\\
\label{eq:align2}
a_{21}&=b_{21}, & a_{22}&=b_{22}+c^2
\end{align}
flalign
\begin{flalign}
\label{eq:flalign1}
a_{11}&=b_{11}, & a_{12}&=b_{12}\\
\label{eq:flalign2}
a_{21}&=b_{21}, & a_{22}&=b_{22}+c^2
\end{flalign}

Плаваючі об'єкти десь тут починаються...

\begin{figure}[htbp]
\emph{Тут буде другий малюнок! Дуже гарний! Кольоровий!}
\caption{Назва малюнка 2}\label{fig:figure2}
\end{figure}

\begin{table}[htbp]
\caption{Назва таблиці 1}\label{tab:table1}
\emph{Тіло першої таблиці}
\end{table}

\end{yafea}
%%%%%%%%%%%%%%%%%%%%%%%%%%%%%%%%%%%%%%%%%%%%%%%%%%%%%%%%%%%%%%%%%%%%%%%%

EXCERPT ENDS HERE.\hrulefill

\begin{equation}\label{eq:aref.intro2}
f(x)=a_0+a_1x+a_2x^2
\end{equation}

\begin{theorem}\label{thm:aref.intro2}
Finish finish finish finish finish.
\end{theorem}

\end{document}
